\clearpage
\section{Experiment}
\label{sec:experiment}

To answer the question how competetive sustainable modes of transport are in comparison to the traditional mode of travel by car, we run an experiment in the city of Cologne.
We do so by calculating the X-minute city metric for hexagons all over Cologne.


\subsection{Data}
\label{subs:data}

\subsubsection{Data Collection}
\label{subs:data_collection}

Our tool requires several datasets as an input: public transport schedules represented by GTFS files, street networks through OSM files, and data that represents free-floating vehicle sharing.
Notably, both GTFS and OSM data can be accessed publicly and can be easily explored, downloaded and preprocessed with the help of our tool.

For GTFS data, we rely on the Mobility Database \shortcite{MobilityDatabase}.
This database serves as an open-source repository containing links to publicly available GTFS feeds globally, standing as the subsequent version of TransitFeeds \shortcite{OpenMobilityDataPublicTransit}.

To use OSM data in practice various tools and services have been developed.
Among these we use, pyrosm \shortcite{Pyrosm} which is a Python library designed specifically for reading OSM data in different formats and conducting data processing operations.
Through pyrosm, we can automatically fetch data from sources like Geofabrik \cite{GeofabrikDownloadServer} and BBBike \cite{BBBikeExtractsOpenStreetMap}, which are two of the most popular OSM data providers.

Using the combination of these resources, our tool ensures easy access to up-to-date GTFS and OSM data.
This allows for easy reproducibility of our results, as well as, the possibility to use our tool for other cities.

\subsubsection{Data Preperation}
\label{subs:data_preperation}

Our tool is able to trim GTFS data to a specific bounding box.
This is especially useful for country-size GTFS feeds.

The GTFS data is also cleaned and converted into a format that is more suitable for RAPTOR.

Specifically, there are two major incompatibilities between the GTFS specification and RAPTOR's notion of routes and trips.
Firstly, each trip belonging to a single route in RAPTOR visits the same stops in the same order.
It is not possible that a trip skips some stops that another trip of the same route visits, much less use a completely different sequence of routes.
In GTFS routes do allow that, as they are much more a group of trips that is presented to the rider under the same name or identifier.
Secondly, GTFS trips allow visiting the same stop multiple times, which is not allowed in RAPTOR.

To overcome these difference our tool splits up routes into smaller routes, that follow the same sequence of stops.
Additionally, it also removes circular trips, altogether.

Similarly, our tool is also able to extract an actual graph from the OSM network.
To do so it utilizes pyrosm.
After extracting the graph from the OSM network, the graph is trimmed to the convex hull of the GTFS stop extended by a small buffer zone.
As a last cleaning step, we remove all nodes, that are not part of the largest weakly connected component.
A weakly connected component is a subgraph in which, if all directed edges were treated as undirected, any two vertices from the subgraph would be connected.
Multiple weakly connected components in graphs derived from OSM data, mostly happen at the border of the considered area and can be neglected.

