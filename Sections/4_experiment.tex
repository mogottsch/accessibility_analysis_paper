\clearpage
\section{Experiment}
\label{sec:experiment}

To answer the question of how competitive sustainable modes of transport are in comparison to the traditional mode of travel by car, we run an experiment in the city of Cologne.
We do so by calculating the X-minute city metric for hexagons all over Cologne.

As we want to compare different modes of transport, we will calculate the metric multiple times for different scenarios, each with a different combination of modes of transport.

\subsection{Scenarios}
\label{subs:scenarios}

No matter, the mode of transport, we always allow for walking, for two reasons.
First, walking is the most accessible mode of transport, available to almost everyone and without any additional costs.
Second, most other modes of transport require walking at some point, be it to the next bus stop or to the next available bicycle.

The first scenario we consider is our baseline scenario, which only considers walking.
This scenario measures what is possible without any additional infrastructure. 
Distinct from other scenarios, it does not require any additional cost, thus presenting the most basic form of urban mobility.

Building on this, the second scenario we consider is the scenario of only walking. 
We consider this scenario as the benchmark scenario, as we hope to achieve similar (or even better) results with more sustainable modes of transport.
Therefore, we use it to answer the question of how competitive sustainable modes of transport are in comparison to the traditional mode of travel by car. 

Transitioning from the simplest form of mobility, the third scenario, focused on public transport, becomes essential to understand the effectiveness and accessibility of urban transit systems. 
This scenario evaluates how well-connected and time-efficient public transportation networks are, and their role in reducing reliance on personal vehicles. 
It also investigates the impact of public transport on urban mobility and its potential in contributing to a more sustainable urban environment. 
Specifically, it assesses whether public transport is a viable alternative to the personal car and whether it actually offers significant advantages over walking, considering the X-minute city metric.

Next, in the fourth scenario, we shift our focus to the dynamics of bicycle sharing systems. 
This scenario is important for assessing the feasibility and attractiveness of cycling as a primary mode of transportation in urban areas. 
We will directly compare it to the public transport scenario, to understand which sustainable mode of transport is superior.

Finally, the fifth scenario combines public transport and bicycle sharing, offering insights into the synergy between these two modes of transport. 
This integrated approach mirrors a growing trend in urban mobility solutions, where multi-modal transport options are increasingly favored. 
It underscores how this combination can bridge the gaps in accessibility and efficiency found when each mode is used independently. 
This scenario is expected to be the most competitive against cars, offering a comprehensive and sustainable urban transit model that could reshape the landscape of city mobility.

We summarize the scenarios in Table \ref{table:scenarios}.

\begin{table}[h]
\centering
\begin{tabular}{|c|p{5cm}|p{5cm}|}
\hline
\textbf{\#} & \textbf{Scenario} & \textbf{Key Points} \\
\hline
1 & Walking & Baseline scenario \\
\hline
2 & Personal Car & Benchmark scenario \\
\hline
3 & Public Transport & Evaluate the effectiveness public transport systems \\
\hline
4 & Bicycle Sharing & Evaluate the effectiveness of bicycle sharing systems \\
\hline
5 & Public Transport and Bicycle Sharing & Evaluate the effectiveness of sustainable multi-modal transport systems \\
\hline
\end{tabular}
\caption{Scenarios for Urban Mobility Analysis}
\label{table:scenarios}
\end{table}

\subsection{Data}
\label{subs:data}

\subsubsection{Data Collection}
\label{subs:data_collection}

Our tool requires several datasets as an input: public transport schedules represented by GTFS files, street networks through OSM files, and data that represents free-floating vehicle sharing.
Notably, both GTFS and OSM data can be accessed publicly and can be easily explored, downloaded and preprocessed with the help of our tool.

For GTFS data, we rely on the Mobility Database \shortcite{MobilityDatabase}.
This database serves as an open-source repository containing links to publicly available GTFS feeds globally, standing as the subsequent version of TransitFeeds \shortcite{OpenMobilityDataPublicTransit}.

To use OSM data in practice various tools and services have been developed.
Among these we use, pyrosm \shortcite{Pyrosm} which is a Python library designed specifically for reading OSM data in different formats and conducting data processing operations.
Through pyrosm, we can automatically fetch data from sources like Geofabrik \cite{GeofabrikDownloadServer} and BBBike \cite{BBBikeExtractsOpenStreetMap}, which are two of the most popular OSM data providers.

Using the combination of these resources, our tool ensures easy access to up-to-date GTFS and OSM data.
This allows for easy reproducibility of our results, as well as, the possibility to use our tool for other cities.

\subsubsection{Data Preperation}
\label{subs:data_preperation}

Our tool is able to trim GTFS data to a specific bounding box.
This is especially useful for country-size GTFS feeds.

The GTFS data is also cleaned and converted into a format that is more suitable for RAPTOR.

Specifically, there are two major incompatibilities between the GTFS specification and RAPTOR's notion of routes and trips.
Firstly, each trip belonging to a single route in RAPTOR visits the same stops in the same order.
It is not possible that a trip skips some stops that another trip of the same route visits, much less use a completely different sequence of routes.
In GTFS routes do allow that, as they are much more a group of trips that is presented to the rider under the same name or identifier.
Secondly, GTFS trips allow visiting the same stop multiple times, which is not allowed in RAPTOR.

To overcome these difference our tool splits up routes into smaller routes, that follow the same sequence of stops.
Additionally, it also removes circular trips, altogether.

Similarly, our tool is also able to extract an actual graph from the OSM network.
To do so it utilizes pyrosm.
After extracting the graph from the OSM network, the graph is trimmed to the convex hull of the GTFS stop extended by a small buffer zone.
As a last cleaning step, we remove all nodes, that are not part of the largest weakly connected component.
A weakly connected component is a subgraph in which, if all directed edges were treated as undirected, any two vertices from the subgraph would be connected.
Multiple weakly connected components in graphs derived from OSM data, mostly happen at the border of the considered area and can be neglected.

