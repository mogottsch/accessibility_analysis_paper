\clearpage
\section{Introduction}
\label{sec:introduction}

%the introduction should be max. 3 pages

% --- CLIMATE CHANGE & SUSTAINABILITY ---

% more needs to be done in order to reach the goals of the Paris Agreement \cite{mitchellMyriadChallengesParis2018}.
% in order to reach the goals of the Paris Agreement emission reduction needs to start immediately \cite{krieglerPathwaysLimitingWarming2018}.
% say that it is unlikely that countries will reach the goals of the paris agreement \cite{liuCountrybasedRateEmissions2021}.
% 72\% of global transport emissions are from road vehicles \cite{Sims2014Transport}.

Climate change poses a significant threat to our planet, and reducing emissions is crucial in addressing this global challenge.
The Paris Agreement, with its ambitious goals to reduce global warming, serves as a call to action.
However, current trends suggest that without immediate action to reduce emissions, achieving these goals is unlikely \cite{krieglerPathwaysLimitingWarming2018, liuCountrybasedRateEmissions2021}.
61.8\% of global emissions in 2015 came from cities, and predictions for 2100 estimate the share to exceed 80\% by 2100 \cite{gurneyGreenhouseGasEmissions2021}.
% I need a citation that says that cities contribute majorly to global emissions
The connection between the large share of city emissions and climate change has led to a critical examination of urban planning and transportation.
In response to the urgent need to reduce emissions, the concept of emission-free cities has emerged as a pivotal strategy.
Emission-free cities aim to create sustainable environments that minimize the carbon footprint, promote the health of residents, and align with the global efforts to mitigate climate change.
Transitioning to these cities is a proactive step toward sustainable living and securing a healthier future for our urban spaces.
With vehicles on the roads accounting for 72\% of all transport-related emissions \cite{Sims2014Transport}, it is clear that urban transportation is a key area to reduce emissions.

% ---- ACCESSIBILTY-BASED PLANNING ----

In order to reduce the amount of car traffic, cities need to be planned in a way that allows people to access everything they need with alternative, more environmentally sustainable modes of transport.
Therefore, a recent trend in urban planning is accessibility-based planning \cite{proffittAccessibilityPlanningAmerican2019} \cite{geursAccessibilityEvaluationLanduse2004a}.
% ---- ACCESSIBILTY METRICS ----
In order for practitioners to be able to plan cities in an accessibility-based way, they need to be able to measure accessibility.

% ---- 15 MINUTE CITIES ----
A modern way of measuring accessibility is to use the X-minute city metric, which is inspired by the 15-minute city concept, which recently gained traction during the COVID pandemic \cite{morenoIntroducing15MinuteCity2021}.

The concept of the 15-minute city is that all the things a person needs to live a good life should be accessible within 15 minutes of walking or cycling.
Traditionally, the modes of transport don't include public transportation or vehicle sharing systems.

% ---- MULTI MODAL ----
However, we argue that in order fully grasp the potential of sustainable transport, it's essential to incorporate all modes, not merely a subset, in the measurement of accessibility

We therefore develop a new tool for accessibility-based planning that incorporates all modes of travel (multi-modal, unrestricted inter-modal).
In addition, this tool will use a metric based on the concept of the 15-minute city, extending it to include all modes of transport and therefore presenting a more holistic view on accessibility via sustainable modes of transport.

We test our tool on the City of Cologne.

% multi-objective & equality stuff is kind of missing
%- (hopefully) give recommendations to practitioners on how to enable equal & fair accessibility (multi-objective)

The remainder of this paper is structured as follows.
In Section \ref{sec:related_work}, we present related work on accessibility-based planning, the 15-minute city concept, and routing algorithms.
Next, in Section \ref{sec:method}, we describe the specifics of our tool, including the data collection process, the routing algorithm, and the accessibility metric.
After that, we introduce our experiment in Section and it's results in Section.
Finally, we conclude with a discussion and conclusion in Section and, respectively.

