\clearpage
\section{Introduction}
\label{sec:introduction}

%the introduction should be max. 3 pages

% --- CLIMATE CHANGE & SUSTAINABILITY ---

more needs to be done in order to reach the goals of the Paris Agreement \cite{mitchellMyriadChallengesParis2018}.
in order to reach the goals of the Paris Agreement emission reduction needs to start immediately \cite{krieglerPathwaysLimitingWarming2018}.
say that it is unlikely that countries will reach the goals of the paris agreement \cite{liuCountrybasedRateEmissions2021}.
72\% of global transport emissions are from road vehicles \cite{Sims2014Transport}.

% ---- EMISSION FREE TRANSPORTATION IN CITIES----

Cities contribute massively to global emissions (citation)
Therefore, transforming cities to be emission free is a key step in the fight against climate change.

% ---- ACCESSIBILTY-BASED PLANNING ----

In order to reduce the amount of car traffic, cities need to be planned in a way that makes allows people to access everything they need with alternative modes of transport.
Therefore, a recent trend in urban planning is accessibility-based planning \cite{proffittAccessibilityPlanningAmerican2019} \cite{geursAccessibilityEvaluationLanduse2004a}.
% ---- ACCESSIBILTY METRICS ----
In order for practitioners to be able to plan cities in an accessibility-based way, they need to be able to measure accessibility.

% ---- 15 MINUTE CITIES ----
A modern way of measuring accessibility is to use the X-minute city metric, which is inspired by the 15-minute city concept, which recently gained traction during the COVID pandemic \cite{morenoIntroducing15MinuteCity2021}.

The concept of the 15-minute city is that all the things a person needs to live a good life should be accessible within 15 minutes of walking or cycling.
Traditionally, the modes of transport don't include public transportation or vehicle sharing systems.

% ---- MULTI MODAL ----
However, we argue that in order to fully capture the potential of all sustainable modes of transport, they should be considered when measuring accessibility.

We therefore develop a new tool for accessibility-based planning that incorporates all modes of travel (multi-modal, unrestricted inter-modal).
In addition this tool will use the X-minute city metric, extending the concept of the 15-minute city to include all modes of transport and therefore presenting a more holistic view on accessibility via sustainable modes of transport.


% multi-objective & equality stuff is kind of missing
%- (hopefully) give recommendations to practitioners on how to enable equal & fair accessibility (multi-objective)
