\clearpage
\section{Introduction}
\label{sec:introduction}

% --- CLIMATE CHANGE & SUSTAINABILITY ---

Climate change poses a significant threat to our planet, and reducing emissions is crucial in addressing this global challenge.
With its ambitious goals to reduce global warming, the Paris Agreement serves as a call to action.
However, current trends suggest that achieving these goals without immediate action to reduce emissions is unlikely \shortcite{krieglerPathwaysLimitingWarming2018, liuCountrybasedRateEmissions2021}.
61.8\% of global emissions in 2015 came from cities, and predictions estimate the share to exceed 80\% by 2100 \shortcite{gurneyGreenhouseGasEmissions2021}.
% I need a citation that says that cities contribute majorly to global emissions
The connection between the large share of city emissions and climate change has led to a critical examination of urban planning and transportation.
In response to the urgent need to reduce emissions, the concept of emission-free cities has emerged as a pivotal strategy.
Emission-free cities aim to create sustainable environments that minimize the carbon footprint, promote the health of residents, and align with the global efforts to mitigate climate change.
Transitioning to these cities is a proactive step toward sustainable living and securing a healthier future for our urban spaces.

% ---- ACCESSIBILTY-BASED PLANNING ----

With vehicles on the roads accounting for 72\% of all transport-related emissions \shortcite{Sims2014Transport}, it is clear that urban transportation is a crucial aspect to reduce emissions.
To reduce the amount of car traffic, cities need to be planned in a way that allows people to conveniently access everything they need with sustainable, and more environmentally friendly modes of transport.
Therefore, one of the main areas of research in the field of urban planning is accessibility-based planning, which was investigated by many studies, including \shortciteA{proffittAccessibilityPlanningAmerican2019, geursAccessibilityEvaluationLanduse2004a}.

% ---- 15 MINUTE CITIES ---- into multi-modal & needs for our tool
A modern way to rethink urban planning is the 15-minute city concept, which recently gained traction during the COVID pandemic \shortcite{morenoIntroducing15MinuteCity2021}.
The 15-minute city envisions a lifestyle where all the essential components for a fulfilling life are conveniently reachable within a 15-minute walking or cycling radius.
Traditionally, the modes of transport within the 15-minute city studies don't include public transportation or vehicle-sharing systems.
However, we argue that to fully grasp the potential of sustainable transportation systems, it's essential to incorporate all available modes of transport, not merely a subset, in measuring accessibility.
Little research has been done on the impact of sustainable transportation modes, beyond walking and using a personal bicycle, on a city's accessibility within the 15-minute city framework and how these modes interact.
We therefore formulate our first research questions as follows:
\begin{enumerate}
  \renewcommand{\labelenumi}{RQ \theenumi.}
  \item How can bicycle sharing and public transport contribute to a city being a 15-minute city?
  \label{rq:bicycle_pt}
  \begin{enumerate}
    \item In what cases is bicycle sharing or public transport more beneficial?
    \label{rq:bicycle_pt:beneficial}
    \item Is a combination of bicycle sharing and public transport necessary, or are they substitutable?
    \label{rq:bicycle_pt:substitutable}
  \end{enumerate}
\end{enumerate}
RQ \ref{rq:bicycle_pt:substitutable} aims to investigate the interactive effects of these transportation modes and examines how their combined or individual use impacts urban accessibility. 
Previous studies have noted synergistic effects between these modes \shortcite{yangImpactPublicBicyclesharing2018,murphyRoleBicyclesharingCity2015,wagnerPublicTransitRouting2017, fishmanBikeShareSynthesis2013, maBicycleSharingPublic2015}; however, there is a notable gap in research regarding their role specifically within the 15-minute city concept.

% ---- COST ----
Incorporating other modes of transport that potentially are fare-based poses a new challenge.
To use these modes of transport, people need to be able to afford them, which might cause inequality in accessibility.
There has been research that investigates the influence of cost on accessibility \shortcite{conwayGettingCharlieMTA2019, el-geneidyCostEquityAssessing2016, guzmanAccessibilityAffordabilityAddressing2017, cuiFullCostAccessibility2018}, but not in the domain of the 15-minute city concept.
Therefore, we formulate our second research question as follows:

\begin{enumerate}
  \renewcommand{\labelenumi}{RQ \theenumi.}
  \setcounter{enumi}{1}
  \item How does cost affect accessibility in the perspective of the 15-minute city, and how can practitioners enable more equal and fair mobility?
  \label{rq:cost_accessibility}
\end{enumerate}

To answer these questions, we develop a new method for accessibility-based planning that incorporates an arbitrary amount of modes of travel.
This method will compute a metric based on the concept of the 15-minute city, extending it to include all modes of transport and, therefore, presenting a more holistic view of accessibility via sustainable modes of transport.
To also consider the interaction between cost and accessibility, our metric will further account for the cost of using the different modes of transport.
To the best of our knowledge, no method exists that can compute accessibility metrics for an arbitrary combination of different transport modes while computing a metric with multiple objectives, like cost and time.
Lastly, we apply our method to the city of Cologne to retrieve actual data.
With that, we formulate our third and fourth research questions as follows:

\begin{enumerate}
  \renewcommand{\labelenumi}{RQ \theenumi.}
  \setcounter{enumi}{2}
  \item How can we measure accessibility in a way that incorporates an arbitrary combination of transport modes while considering the associated costs?
  \label{rq:measure_accessibility}
  \item What are specific recommendations for urban planning in Cologne?
  \label{rq:recommendations}
\end{enumerate}

Our research makes significant contributions both theoretically and practically. Theoretically, our contributions are threefold:
Theoretically, we introduce a new method for accessibility-based urban planning that incorporates various transport modes and considers the cost of using these modes.
Our second contribution is the development of a novel metric rooted in the 15-minute city concept, offering a fresh perspective on assessing urban accessibility.
Thirdly, we analyze the influence and interaction between multiple transportation modes within the 15-minute city framework and examine how cost factors impact accessibility.
Practically, our work yields a significant tool for urban planners. 
This tool facilitates the planning of cities with a focus on accessibility, aligning closely with the 15-minute city concept. 
It offers a pragmatic approach to urban design, enabling planners to create more accessible, efficient, cost-effective urban environments.

The remainder of this paper is structured as follows.
Section \ref{sec:related_work} presents related work on accessibility-based planning, the 15-minute city concept, and routing algorithms.
Then, in Section \ref{sec:method}, we describe the specifics of our approach, including the data collection process, our routing algorithm, and our accessibility metric.
After that, we introduce the use case on which we will apply our method in Section \ref{sec:experiment} and present the results in Section \ref{sec:results}.
Finally, we conclude with our discussion Section \ref{sec:discussion}.

