%%%%%%%%%%%%%%%%%%%%%%%%%%%%%%%%%%%%%%%%%%%%%%%%%%%%%%%%%%%% 
% This is the official template for theses and seminar papers from the Chair for Information Systems for Sustainable Society (IS3) at the University of Cologne

%
%PREAMBLE
%%%%%%%%%%%%%%%%%%%%%%%%%%%%%%%%%%%%%%%%%%%%%%%%%%%%%%%%%%%%%

\documentclass[a4paper, twoside, 12pt]{article}
\usepackage[utf8]{inputenc}
\usepackage[T1]{fontenc}
\usepackage{graphicx}
\usepackage{longtable}
\usepackage{hyperref}
\usepackage{caption}

% set margins for double-sided printing \usepackage[left=2.5cm, right=2.5cm, top=2.5cm, bottom=2.5cm, bindingoffset=1.5cm, head=15pt]{geometry} \usepackage{setspace}
% set headers
\usepackage{fancyhdr}
\pagestyle{fancy}
\fancyhead{}
\fancyfoot{}
\fancyhead[LE,RO]{\textsl{\leftmark}}
\fancyhead[RE,LO]{\thesisauthor}
\fancyfoot[C]{\thepage}
\renewcommand{\headrulewidth}{0.4pt}
\renewcommand{\footrulewidth}{0pt}

\setlength{\headheight}{14.5pt} % fixes headheight warning

% set APA citation style
\usepackage{apacite}
\usepackage[numbib,notlof,notlot,nottoc]{tocbibind}
\pagenumbering{gobble}

\usepackage{amsfonts}
\usepackage{amsmath}

\usepackage{caption}
\usepackage{eurosym}
\usepackage{subcaption}
\usepackage{siunitx}

%%%%%%%%%%%%%%%%%%%%%%%%%%%%%%%%%%%%%%%%%%%%%%%%%%%%%%%%%%%%%
%THESIS Parameters 
%%%%%%%%%%%%%%%%%%%%%%%%%%%%%%%%%%%%%%%%%%%%%%%%%%%%%%%%%%%%%

\title{Extending Accessibility Analysis With True Multi-Modality}

\newcommand{\thesisdate}{January 01, 2019}
\newcommand{\thesisauthor}{Moritz Gottschling} %input name
\newcommand{\studentID}{7350270} %input student ID
\newcommand{\thesistype}{Master Thesis} % Set either to Bachelor or Master
\newcommand{\supervisor}{Univ.-Prof. Dr. Wolfgang Ketter}
\newcommand{\cosupervisor}{Philipp Peter}

%%%%%%%%%%%%%%%%%%%%%%%%%%%%%%%%%%%%%%%%%%%%%%%%%%%%%%%%%%%%%
%DOCUMENT
%%%%%%%%%%%%%%%%%%%%%%%%%%%%%%%%%%%%%%%%%%%%%%%%%%%%%%%%%%%%%

\begin{document}

%%%%%%%%%%%%%%%%%%%%%%%%%%%%%%%%%%%%%%%%%%%%%%%%%%%%%%%%%%%%%
%TITLE PAGE (Pre-defined, just change parameters above)
%%%%%%%%%%%%%%%%%%%%%%%%%%%%%%%%%%%%%%%%%%%%%%%%%%%%%%%%%%%%%
\input{Template/Title.tex}

%%%%%%%%%%%%%%%%%%%%%%%%%%%%%%%%%%%%%%%%%%%%%%%%%%%%%%%%%%%%%
%SOOA
%%%%%%%%%%%%%%%%%%%%%%%%%%%%%%%%%%%%%%%%%%%%%%%%%%%%%%%%%%%%%
\input{Template/SOOA.tex}

%%%%%%%%%%%%%%%%%%%%%%%%%%%%%%%%%%%%%%%%%%%%%%%%%%%%%%%%%%%%%
%ABSTRACT
%%%%%%%%%%%%%%%%%%%%%%%%%%%%%%%%%%%%%%%%%%%%%%%%%%%%%%%%%%%%%
\clearpage
\thispagestyle{empty}
\section*{Abstract}

This thesis addresses the critical issue of high emissions from urban traffic and the transformation of cities into emission-free environments, focusing on the under-researched area of multi-modal accessibility analysis within the 15-minute city concept. 
The research explores the roles of bicycle sharing and public transport in realizing a 15-minute city, assessing their effectiveness and interplay, and examining the impact of cost on accessibility.
Methodologically, the research introduces a novel location-based accessibility metric grounded in the 15-minute city concept, which integrates multiple modes of transport and cost considerations.
We develop a unique routing algorithm incorporating multiple objectives and transport modes that can be easily exchanged and extended.
It is then applied to Cologne to gather relevant data and conduct a comprehensive analysis.
The results indicate that while bicycle sharing and public transport enhance urban accessibility, they do not match cars' effectiveness.
However, bicycle sharing generally proves more effective, particularly in central areas, whereas public transport is more beneficial in remote regions with lower accessibility.
The findings also reveal that these modes are largely non-substitutable. 
Practically, the research offers urban planners a valuable tool, aligning with the 15-minute city concept to facilitate more accessible, efficient, and cost-effective urban designs, thereby making significant theoretical and practical contributions to urban planning.


%%%%%%%%%%%%%%%%%%%%%%%%%%%%%%%%%%%%%%%%%%%%%%%%%%%%%%%%%%%%%
%TOC,TOF,TOT
%%%%%%%%%%%%%%%%%%%%%%%%%%%%%%%%%%%%%%%%%%%%%%%%%%%%%%%%%%%%%
\clearpage
\pagenumbering{Roman}
\tableofcontents
\clearpage
\listoffigures
\clearpage
\listoftables
\clearpage

\pagenumbering{arabic}


%%%%%%%%%%%%%%%%%%%%%%%%%%%%%%%%%%%%%%%%%%%%%%%%%%%%%%%%%%%%%
%MAIN PART
%%%%%%%%%%%%%%%%%%%%%%%%%%%%%%%%%%%%%%%%%%%%%%%%%%%%%%%%%%%%%

\clearpage
\section{Introduction}
\label{sec:introduction}

%the introduction should be max. 3 pages

% --- CLIMATE CHANGE & SUSTAINABILITY ---

% more needs to be done in order to reach the goals of the Paris Agreement \cite{mitchellMyriadChallengesParis2018}.
% in order to reach the goals of the Paris Agreement emission reduction needs to start immediately \cite{krieglerPathwaysLimitingWarming2018}.
% say that it is unlikely that countries will reach the goals of the paris agreement \cite{liuCountrybasedRateEmissions2021}.
% 72\% of global transport emissions are from road vehicles \cite{Sims2014Transport}.

Climate change poses a significant threat to our planet, and reducing emissions is crucial in addressing this global challenge.
The Paris Agreement, with its ambitious goals to reduce global warming, serves as a call to action.
However, current trends suggest that without immediate action to reduce emissions, achieving these goals is unlikely \cite{krieglerPathwaysLimitingWarming2018, liuCountrybasedRateEmissions2021}.
61.8\% of global emissions in 2015 came from cities, and predictions for 2100 estimate the share to exceed 80\% by 2100 \cite{gurneyGreenhouseGasEmissions2021}.
% I need a citation that says that cities contribute majorly to global emissions
The connection between the large share of city emissions and climate change has led to a critical examination of urban planning and transportation.
In response to the urgent need to reduce emissions, the concept of emission-free cities has emerged as a pivotal strategy.
Emission-free cities aim to create sustainable environments that minimize the carbon footprint, promote the health of residents, and align with the global efforts to mitigate climate change.
Transitioning to these cities is a proactive step toward sustainable living and securing a healthier future for our urban spaces.
With vehicles on the roads accounting for 72\% of all transport-related emissions \cite{Sims2014Transport}, it is clear that urban transportation is a key area to reduce emissions.

% ---- ACCESSIBILTY-BASED PLANNING ----

In order to reduce the amount of car traffic, cities need to be planned in a way that allows people to conveniently access everything they need with alternative, more environmentally sustainable modes of transport.
Therefore, a recent trend in urban planning is accessibility-based planning \cite{proffittAccessibilityPlanningAmerican2019} \cite{geursAccessibilityEvaluationLanduse2004a}.
% ---- ACCESSIBILTY METRICS ----
In order for practitioners to be able to plan cities in an accessibility-based way, they need to be able to measure accessibility.

% ---- 15 MINUTE CITIES ---- into multi modal & needs for our tool
A modern way of measuring accessibility is to use the X-minute city metric, which is inspired by the 15-minute city concept, which recently gained traction during the COVID pandemic \cite{morenoIntroducing15MinuteCity2021}.
The concept of the 15-minute city is that all the things a person needs to live a good life should be accessible within 15 minutes of walking or cycling.
Traditionally, the modes of transport don't include public transportation or vehicle sharing systems.
However, we argue that in order fully grasp the potential of sustainable transport, it's essential to incorporate all modes, not merely a subset, in the measurement of accessibility.
Little research has been done on how sustainable modes of transport, other than walking and cycling with a personal bicycle, contribute to the accessibility of a city in the context of the 15-minute city and how those modes interact with each other.
We therefore formulate our first research questions as follows:

\begin{itemize}
  \item How can bicycle sharing and/or public transport contribute to a city being a 15-minute city?
  \begin{itemize}
    \item In what cases is bicycle sharing or public transport more beneficial?
    \item Is a combination of bicycle sharing and public necessary or are they substitutable?
  \end{itemize}
\end{itemize}

% ---- COST ----
However, incorporating other modes of transport, that potentially are fare-based, poses a new challenge.
In order to be able to use these modes of transport, people need to be able to afford them, which might cause inequality in accessibility.
There has been research that investigates the influence of cost on accessibility \shortcite{conwayGettingCharlieMTA2019, el-geneidyCostEquityAssessing2016, guzmanAccessibilityAffordabilityAddressing2017, cuiFullCostAccessibility2018}, but not in the domain of the 15-minute city concept.
Therefore, we formulate our second research question as follows:

\begin{itemize}
  \item How does cost affect accessibility in the perspective of the 15-minute city, and how can practitioners enable more equal and fair mobility?
\end{itemize}

To answer these questions, we develop a new method for accessibility-based planning that incorporates an arbitrary amount of modes of travel.
In addition, this method will incorporate a metric based on the concept of the 15-minute city, extending it to include all modes of transport and therefore presenting a more holistic view on accessibility via sustainable modes of transport.
To also consider the interaction between cost and accessibility, our metric will also account for the cost of using the different modes of transport.
To the best of our knowledge, there exists no method that is able to compute accessibility metrics for an arbitrary amount of modes of transport and also considers the cost of using these modes.
Lastly, to retrieve actual data, we apply our method to the city of Cologne.
With that we formulate our third and fourth research questions as follows:

\begin{itemize}
  \item How can we measure accessibility in a way that incorporates an arbitrary amount of modes of transport and also considers the cost of using these modes?
  \item What are specific recommendations for urban planning in Cologne?
\end{itemize}

Our theoretical contributions are two-fold.
Firstly, we present a new method for accessibility-based planning that incorporates an arbitrary amount of modes of transport and also considers the cost of using these modes.
Secondly, we formulate a new metric based on the concept of the 15-minute city.
Practically, we contribute a tool that can be used by urban planners to plan cities in an accessibility-based way, closely aligned with the 15-minute city concept.

The remainder of this paper is structured as follows.
In Section \ref{sec:related_work}, we present related work on accessibility-based planning, the 15-minute city concept, and routing algorithms.
Next, in Section \ref{sec:method}, we describe the specifics of our method, including the data collection process, the routing algorithm, and the accessibility metric.
After that, we introduce the case example, we will apply our method on in Section \ref{sec:experiment} and present the results in Section \ref{sec:results}.
Finally, we conclude with our discussion Section \ref{sec:discussion}.


\clearpage
\section{Related Work}
\label{sec:related_work}

Urban planning and transportation is evolving rapidly due to the increasing emphasis on sustainability and proximity-based accessibility. 
This section consists of a comprehensive literature review focusing on critical areas such as accessibility analysis, routing algorithms, and integrating public transport data with street network data. 
The objective is to build a theoretical foundation for developing methodologies to optimize urban accessibility. 
The first part of this section explores the paradigm shift from traditional mobility-based to accessibility-based planning.
This shift is crucial in addressing modern urban challenges such as reducing greenhouse gas emissions and promoting sustainable transportation. 
Subsequently, we examine various routing algorithms central to assessing accessibility effectively. 
These include some well-established algorithms like Dijkstra, Multi-Label-Correcting (MLC), and innovative approaches tailored for public transport networks, such as RAPTOR, McRAPTOR, and Multi-modal Multicriteria RAPTOR (MCR). 
Furthermore, the integration of public transport data, primarily through the General Transit Feed Specification (GTFS), and street network data sourced from OpenStreetMap (OSM) are discussed. 
This integration is vital for a realistic simulation of urban dynamics and enables the reproducibility of our results. 
Finally, we address the significance of multimodality and intermodality in urban planning.
This is important because considering diverse transport modes, including bicycle sharing and public transport, and investigating their synergies are vital to retrieving a holistic view of urban accessibility.
This literature review sets the stage for the subsequent sections, where we develop and test a new tool for accessibility-based planning, incorporating multi-modal and intermodal transport options.

\subsection{Accessibility-Based Planning}
\label{subsec:accessibility_based_planning}

Traditional mobility-based planning primarily focuses on reducing congestion and facilitating movement, often prioritizing automobile travel \shortcite{proffittAccessibilityPlanningAmerican2019}.
However, this approach is becoming increasingly outdated, failing to address modern challenges like reducing greenhouse gas emissions.
This is where accessibility-based planning comes into play, which focuses on planning cities in a way that provides residents with fast access to all essential services \shortcite{proffittAccessibilityPlanningAmerican2019}.
Having easy access to all essential services has the potential to reduce the need for car ownership and its frequent usage.
This shift foregrounds accessibility as the primary goal rather than merely focusing on mobility. 
Recognizing that mobility is a means to an end and not the ultimate objective in urban planning. 

\shortciteA{geursAccessibilityEvaluationLanduse2004a} describe four types of accessibility-based planning, seen in Table \ref{tab:accessibility_measures}.
Infrastructure-based accessibility is an approach that evaluates the performance and service level of transportation infrastructure.
It involves an analysis of various factors, such as the levels of congestion and average travel speeds. 
The primary goal of this approach is to optimize the physical layout and capacity of transport networks to improve access to various destinations.
Even though \shortciteA{geursAccessibilityEvaluationLanduse2004a} describe this as an accessibility-based approach, it more closely resembles the traditional mobility-based approach, as it focuses on the infrastructure itself rather than accessibility.
Location-Based or Place-Based Accessibility focuses on the accessibility of essential services or employment opportunities within specific travel time or cost limits. 
This method calculates the number of critical destinations, such as schools, hospitals, or workplaces that are reachable within a reasonable commute. It emphasizes the spatial distribution of resources and services.
Person-Based Accessibility centers on the individual's experience in accessing various activities. This perspective acknowledges that accessibility can vary significantly based on personal factors like age, income, or physical ability. It aims to tailor urban planning to address diverse personal needs and constraints, considering personal travel patterns and individual circumstances.
Utility-based accessibility evaluates the economic benefits of transportation investments. 
This approach focuses on the returns of such investments regarding improved accessibility and their overall utility to the community. 
It includes considerations of how transportation improvements can enhance residents' quality of life and economic opportunities.

\begin{table}[ht]
\centering
\caption{Categories of Accessibility-Based Planning Measures}
\label{tab:accessibility_measures}
\begin{tabular}{|l|l|l|}
\hline
\textbf{Category} & \textbf{Focus} & \textbf{Examples} \\ 
\hline
Infrastructure-Based & 
  \begin{tabular}[c]{@{}l@{}}
    - Traffic performance \\
    analysis \\ 
    - Service level of \\
    infrastructure
  \end{tabular} & 
  \begin{tabular}[c]{@{}l@{}}
    - Level of congestion \\ 
    - Average travel speed
  \end{tabular} \\ 
\hline
  \begin{tabular}[c]{@{}l@{}}
      Location-Based \\
      or Place-Based
  \end{tabular} & 
  \begin{tabular}[c]{@{}l@{}}
    - Level of accessibility\\
    to locations \\ 
    - w.r.t. time \& cost
  \end{tabular} & 
  \begin{tabular}[c]{@{}l@{}}
    - Number of jobs within\\
    30 minutes \\
  \end{tabular} \\ 
\hline
Person-Based & 
  \begin{tabular}[c]{@{}l@{}}
    - Individual travel time \\ 
  \end{tabular} & 
  \begin{tabular}[c]{@{}l@{}}
    - Individual's travel time \\ 
      between activities
  \end{tabular} \\ 
\hline
Utility-Based & 
  \begin{tabular}[c]{@{}l@{}}
    - Economic benefits \\ 
  \end{tabular} & 
  \begin{tabular}[c]{@{}l@{}}
    - Transportation investments \\ 
    returns
  \end{tabular} \\ 
\hline
\end{tabular}
\end{table}


\subsection{15-Minute City}
\label{subsec:15_minute_city}


% ----INTRO OF 15 MINUTE CITY ----
A recent stream of research in the field of urban planning is the concept of the 15-minute city.
The concept is based on the idea that all essential services and amenities should be reachable within a 15-minute walking radius, focusing on reducing travel times and enhancing urban livability.
Even though it is a concept rather than a concrete metric, the 15-minute city is closely related to location-based planning measures.
The 15-minute city concept was first introduced by Carlos Moreno, a professor at the Sorbonne University in Paris \shortcite{morenoIntroducing15MinuteCity2021}.
It was popularized by the mayor of Paris, Anne Hidalgo, who made it a central part of her re-election campaign \shortcite{gongadzeParisVision15Minute2023}.
% \cite{willberg15minuteCityAll2023} say that it is crucial to consider elderly people who cannot walk as fast as younger people.
% We could use that as an argument as to why public transport needs to be considered.

% advantages
% cultivate stronger social relationships because people are more likely to meet \cite{allamProximityBasedPlanning15Minute2020}
% a kind of social distancing, because of primarily traveling by walking or cycling \cite{allamProximityBasedPlanning15Minute2020}
% a more environmentally sustainable mode of transport in cities, which contributes to SDG 11 \& 13 \cite{allamProximityBasedPlanning15Minute2020} \cite{papasUrbanMobilityEvolution2023}
% advantages
The 15-minute city model offers several advantages instrumental in shaping modern urban environments.
Primarily, it cultivates stronger social relationships \shortcite{allamProximityBasedPlanning15Minute2020}.
Residents are more likely to encounter each other regularly in a city where essential amenities are within a 15-minute walking or cycling radius. 
This frequent interaction fosters community bonds and a sense of belonging, which benefits urban livability.
Additionally, this model inevitably promotes a unique form of social distancing. 
As residents primarily travel by walking or cycling, there's a natural reduction in gathering large crowds, typically seen in mass transit systems. 
This not only aids in maintaining public health standards but also diminishes the stress often associated with crowded urban transportation \shortcite{allamProximityBasedPlanning15Minute2020}.
Moreover, the shift towards walking and cycling represents a more environmentally sustainable mode of urban transport. 
This transition aligns closely with Sustainable Development Goals 11 and 13, which advocate for sustainable cities and communities, and taking urgent action to combat climate change, respectively. 
By reducing reliance on motorized vehicles, the 15-minute city concept contributes to lowering carbon emissions and minimizing the overall environmental impact of urban areas \shortcite{allamProximityBasedPlanning15Minute2020, papasUrbanMobilityEvolution2023}.

Furthermore, an important economic benefit of the 15-minute city is the significant reduction in traffic. 
This decrease in traffic congestion not only enhances the efficiency of urban transport systems but also leads to economic savings for both individuals and the city as a whole, due to lower transportation costs and time savings.
This aspect underscores the economic viability of the 15-minute city model in creating more efficient and cost-effective urban environments \shortcite{allamProximityBasedPlanning15Minute2020, papasUrbanMobilityEvolution2023}.

In addition, developing cities around the 15-minute city concept can play a crucial role in reducing social inequalities. 
Since walking is a free mode of transport, it becomes an accessible option for all socioeconomic groups, thereby leveling the playing field for urban residents. 
This inclusivity is central to the concept's appeal and effectiveness \shortcite{weng15minuteWalkableNeighborhoods2019b, gustafsonExaminingSpatialChange2022}.
However, when incorporating other fare-based modes of transport, such as bicycle sharing, it is vital to consider the fares and their impact on accessibility and equity. 
This consideration is a crucial reason for adopting multi-objective approaches in urban planning, ensuring that the benefits of the 15-minute city are equitably distributed across all sections of society.


\subsection{Accessibility Metrics Based on the 15-Minute City}
\label{subsec:accessibility_metrics_based_on_the_15_minute_city}

% --- NEXI ---
Quantitative studies have developed various ways to measure how well cities match the 15-minute city concept by developing metrics that encapsulate this principle.  
\shortciteA{olivariAreItalianCities2023} contribute to this research by creating the NExt proXimity Index (NEXI), which has two components: the NEXI-Minutes and the NEXI-Global. 
The NEXI-Minutes looks at the accessibility of various urban amenities, split into categories that range from educational institutions to entertainment venues and grocery stores.
It does so by calculating the time needed to reach the closest facility within each category.
Complementing this, the NEXI-Global, inspired by the Walk Score method \shortcite{WalkScoreMethodology2023}, combines these individual times through a weighted average into an overall score, giving a holistic view of the accessibility of a city.

NEXI is unique because it is both global and local. 
Namely, it is globally applicable due to its reliance on OpenStreetMap data, yet sufficiently detailed to assess local conditions.
This has been proven by applying it all over Italy, where it is showcased on an interactive map through a hexagonal grid that makes it easy to see which areas are doing well and which need improvement.
The significance of the NEXI, as underscored by \shortciteA{olivariAreItalianCities2023}, is its role in enabling data-driven policy-making to develop 15-minute cities. 
Therefore, their index enables accessibility-based planning.
% Despite the potential benefits, \cite{olivariAreItalianCities2023} acknowledge the challenges in realizing the 15-minute city model, notably the substantial investments and strategic planning required. 

However, there are some limitations in the NEXI metrics. 
The NEXI-Minutes offers separate metrics for each category, which may lead to a fragmented understanding of urban accessibility. 
This multi-metric approach can make comprehensive evaluation challenging. 
In addition, the NEXI-Global, while aggregating these categories, introduces complexity through its weighted scoring system. 
The weighted average is hard for humans to interpret, and the score from 0 to 100 disconnects the metric from the intuitive meaning of minutes.
This makes it more difficult for urban planners and policymakers to interpret and utilize the results effectively. 
These factors suggest a need for refinement in the NEXI methodology to enhance its practical utility in urban development and planning.


% --- CITYACCESSMAPS ---
Another study by \shortciteA{nicolettiDisadvantagedCommunitiesHave2023} introduces an accessibility metric that explores the connection between urban infrastructure and social inequality. 
The researchers developed an open, data-driven framework to analyze how different communities within cities access essential services, and they discovered that access to urban amenities like healthcare, education, and transportation is not evenly distributed. 
The study examined over 50 types of amenities across 54 cities worldwide and found a common pattern: in all cities, access to infrastructure followed a log-normal distribution, indicating that a small number of communities have very high access to amenities, while the majority have moderate to low access. 
The log-normal distribution suggests that improvements in accessibility are not proportionately distributed across the urban landscape. 
Instead, there are diminishing returns in accessibility as one moves from the most to the least accessible areas. 
The skew towards lower accessibility for most communities underscores systemic urban planning and infrastructure development issues that disproportionately affect disadvantaged populations.
This pattern was consistent even when considering various socioeconomic factors.


The framework developed by \shortciteA{nicolettiDisadvantagedCommunitiesHave2023} is flexible and adaptable, allowing city planners to tailor it to local needs and priorities. 
It's a tool that can help identify which groups in a city are most affected by inequality concerning accessibility to services. 
Similar to \shortciteA{olivariAreItalianCities2023}, they emphasize the role of open data to guide urban planning and policy.
While \shortciteA{olivariAreItalianCities2023} provide a tool to measure accessibility and leave the analysis and interpretation to practitioners, \shortciteA{nicolettiDisadvantagedCommunitiesHave2023} directly reveal the disparities in accessibility and provide a framework to analyze them.

% --- barcelona
Another metric introduced by \shortciteA{ferrer-ortizBarcelona15MinuteCity2022} calculates whether a specific area has access to a particular category within 15 minutes for Barcelona.
They derive their categories from the six 15-minute city "urban social functions" defined by \shortcite{morenoIntroducing15MinuteCity2021}.
However, they only consider a small subset of four categories: care, education, provisioning, and entertainment.
Their data is taken from a variety of sources, all of which are specific to Spain or Barcelona.
This makes it challenging to apply their method to other cities.
To calculate the distances between the amenities and the areas, they use ESRI's ArcGIS Network Analyst, which is proprietary software, making applying their method to other cities difficult.
Interestingly, the study revealed a high level of service accessibility across Barcelona, with residents having, on average, access to 22 of the 24 services within 15 minutes. 
However, disparities were observed between the city center and suburban areas, indicating spatial inequalities in accessibility.
The findings underscore the critical role of urban planning in ensuring equitable access to amenities and demonstrate that most of Barcelona's areas align with the 15-Minute City ideal. 

% --- outro & transition to next algorithms ---
Meanwhile, the computation of travel time by all previously named authors is based on walking simulations. 
This fails to capture the reality of urban mobility, which consists of various modes of transport, such as public transport, cycling, and driving.
Moreover, the NEXI's fragmented multi-metric approach and complex weighted scoring system can hinder practical interpretation and application.
Acknowledging these shortcomings, we aim to introduce a novel multi-modal accessibility metric, which we present in Section \ref{subsec:metric}.
This proposed metric will encompass a broader spectrum of transportation modes and a more intuitive understanding, providing a more accurate and comprehensive evaluation of urban accessibility for effective urban planning.

\subsection{Routing Algorithms}
\label{subsec:routing_algorithms}

Following the discussion of accessibility metrics and their application in urban planning, particularly in the context of the 15-minute city model, this subsection transitions to a critical component necessary to calculate these metrics: routing algorithms.
Here, the focus is to explore various routing algorithms instrumental in calculating the shortest and most efficient paths within urban environments. 

\subsubsection{Graph-Based Algorithms}

Before exploring the specific algorithms, it is vital to understand the theoretical foundations of graph-based routing algorithms, so we first present the theoretical background.

\paragraph{Theoretical Background}

The primary goal of routing algorithms is to identify the optimal path between a designated origin and a specific destination.
Typically, this is captured using a graph representation:
\[ G = (V, E) \]
where $V$ represents a set of nodes or locations and $E$ encapsulates the set of edges, which correspond to connections between these nodes.

For each edge \( e \in E \), there's an associated weight \( w(e) \in \mathbb{R} \) that characterizes the cost of traversing it.
This cost might be determined by factors such as distance or travel time.
Consequently, the shortest path can be expressed as:
\[ \langle v_0, e_0, v_1, e_1, \dots, v_n \rangle \]
Here, \( v_0 \) denotes the origin, \( v_n \) the destination, and the edges must connect the nodes in the sequence:
\[ e_i = (v_i, v_{i+1}) \quad \text{for} \quad i \in \{0, \dots, n-1\} \]
In accessibility contexts, the primary concern frequently revolves around determining the accumulated cost, \( d(v_n) = \sum_{e \in E} w(e) \), to reach the destination rather than the actual path.

The problem may also encompass multiple objectives, such as considering both time and monetary cost of travel.
Under these circumstances, the edge weight is represented as a vector:
\[ w(e) \in \mathbb{R}^k \]
where \( k \) stands for the total objectives count.
Unlike the more straightforward single-objective case with a singular optimal path, the multi-objective optimization yields a Pareto set, constituting several optimal routes.
A Pareto set refers to a set of solutions that are non-dominated by any other solution.
This means that for each solution in the Pareto set, no other solution is better for all objectives.
The Pareto set represents an optimal trade-off among the different objectives, where improving one aspect would worsen another.
For example, a Pareto set could contain multiple paths, where one path is faster but more expensive, while another is slower but cheaper.

The value of these paths is depicted using a label: \( l \in \mathbb{R}^k \) where \( l_i \in \mathbb{R} \) denotes the value for the \( i \)-th objective.
This label can be considered a multidimensional extension of \( d(v_n) \) from the single-objective scenario.
The Pareto set associated with destination node \( v_n \) is often termed as a bag, expressed as \( B(v_n) \), comprising labels that are not dominated by each other.
Domination is defined as follows: \( l' \) dominates \( l \) if \( l'_i \leq l_i \) for all \( i \in \{1, \dots, k\} \) and \( l'_i < l_i \) for at least one \( i \in \{1, \dots, k\} \).
Intuitively, this means that \(l' \) is at least as good as \( l \) in all objectives and strictly better in at least one objective.

The goal of routing algorithms used in accessibility analysis is finding the distance in the single objective case and the bag in the multi-objective case, often without saving the actual paths.
In accessibility analysis, one is often not interested in finding the optimal path between two nodes but between one node and all other nodes.
The former is known as a one-to-one query, while the latter is called a one-to-all query.

\paragraph{Dijkstra}
\label{subsubsec:dijkstra}
The most straightforward approach to compute the shortest paths in a graph is the Dijkstra algorithm \shortcite{dijkstra1959note}.
Dijkstra's algorithm initiates at a designated start node \( s \in V \) and employs a priority queue to systematically determine the shortest path to each subsequent node \( v \in V \).
Initially, the distance to the start node \( s \) is set to zero, while the distances to all other nodes are set to infinity.
The algorithm dequeues the node \( u \) with the smallest known distance from the priority queue in each iteration.
It then examines each outgoing edge \( e = (u, v) \) from \( u \), updating the distance to \( v \) if a shorter path through \( u \) is discovered.
Specifically, if \( \text{dist}(u) + w(e) < \text{dist}(v) \), then \( \text{dist}(v) \) is updated to \( \text{dist}(u) + w(e) \), and \( v \) is enqueued into the priority queue for future exploration.
The node \( u \) is marked as visited by adding it to the set \( V_{\text{visited}} \).
Depending on the goal, the algorithm terminates when the destination node is dequeued (one-to-one) or when the priority queue is empty (one-to-all).


However, this simple approach has multiple problems.
Firstly, the Dijkstra algorithm is not able to handle multiple criteria.
Secondly, the runtime of Dijkstra's algorithm is \( O(|E| + |V| \log |V|) \), which is slow for large graphs.

\paragraph{MLC}
\label{subsubsec:mlc}

The Multi-Label-Correcting (MLC) \shortcite{hansenBicriterionPathProblems1980} algorithm is an extension of Dijkstra's algorithm to handle multi-objective scenarios.
As mentioned in Section \ref{subsec:routing_algorithms}, in the multi-objective case, we try to find the bag of the destination node.
Specifically, for \(k\) criteria, each node \(v\) retains a bag of \(k\)-dimensional labels. Such a list encapsulates a set of Pareto-optimal paths from the starting node to \(v\).
Similarly to Dijkstra's algorithm, MLC initializes all nodes with an empty bag, except for the start node, which is initialized with a label of \( (0, \dots, 0) \in \mathbb{R}^k \).
Each iteration extracts the lexicographically smallest label instead of selecting the node with the minimum distance.
When a label is extracted and \(v\) is its corresponding node, updates are made for all connected edges \( (v, w) \).
The update process consists of comparing a newly generated tentative label against all labels within the bag of \(w\).
This new label is only inserted into the bag if any existing label does not dominate it.
Conversely, any label now dominated by the new entry is removed.
Each time a label is inserted into a bag, it is also inserted into the priority queue.
The algorithm terminates when the priority queue is empty.

The major drawback of the MLC algorithm is its runtime, which is even slower than Dijkstra's algorithm because each node can be visited multiple times.


\paragraph{Graph-Based Algorithms in Public Transport}
\label{subsubsec:graph_based_algorithms_in_public_transport}


In the context of accessibility analysis, the previously mentioned algorithms can be used directly for walking, cycling, and driving networks.
%To do so, the network is represented as a graph, and the edge weights are set to the travel time.
%It is also possible to represent multi-modal networks as a graph, connecting the different graphs with transfer edges.
%These transfer edges represent
However, public transport networks pose a challenge since they contain time-dependent information, such as the departure time of a trip.
To overcome this challenge, two approaches are commonly used, the time-expanded and the time-dependent approach, as explained by \shortciteA{muller-hannemannTimetableInformationModels2007}.
The idea behind these methods is to artificially craft a graph that represents the public transport network in a way that allows the use of graph-based algorithms.
While enabling the use of graph-based algorithms to solve routing in time-dependent transport networks, both approaches result in massive graphs and therefore suffer from runtime problems. 
Because of these runtime problems, we will not go into detail about them but rather introduce algorithms that take advantage of the specifics of time-dependent transport networks, resulting in much better runtimes.

\subsubsection{Schedule-based Algorithms}

Schedule-based Algorithms are a class of algorithms specifically designed to solve routing problems in time-dependent transport networks.
We start with the most influential algorithm in this class, RAPTOR \shortcite{dellingRoundBasedPublicTransit2015}, and then introduce some of its extensions and other algorithms based on it.

\paragraph{RAPTOR}
\label{subsubsec:raptor}

To overcome the runtime problems of graph-based approaches, \shortciteA{dellingRoundBasedPublicTransit2015} introduce one of the most prominent routing algorithms for public transport, called Round based Public Transit Optimized Router algorithm (RAPTOR). % which is also used by R5
Unlike traditional Dijkstra-based algorithms, RAPTOR operates in rounds, looking at each route (such as a bus line) in the network at most once per round, where one round represents a single trip.

As RAPTOR does not operate on a graph, we first introduce the problem statement.
Raptor operates on a scheduled network consisting of routes \(r\), trips \(t\), stops \(p\), and stop times that associate trips with stops.
% Next paragraph should probably be handled in the method
% Here, it is essential to understand the differences between what we consider a route in RAPTOR and other algorithms and what is considered a route in GTFS data. In GTFS data, a route can have multiple stop sequences, which we will call paths. For example, it is almost always the case that the path and the reversed path are associated with the same route. Therefore, it is not straightforward to convert GTFS routes into routes that can be used in RAPTOR.
A route is associated with a sequence of stops \(stops(r) = \langle p_1, \dots, p_n \rangle\).
A route has multiple trips ordered by their departure time \(trips(r) = \langle t_1, \dots, t_m \rangle\).
One trip associates arrival and departure times with each stop of the route, denoted by \(arrivalTime(t, s) \in \mathbb{N}\) and \(departureTime(t, s) \in \mathbb{N}\) respectively.
% talk about time representation?
Trips of the same route must not overtake each other, formally:
\[departureTime(t_i, p_j) \leq arrivalTime(t_{i+1}, p_j)\]
for all \(i \in \{1, \dots, m-1\}\) and \(j \in \{1, \dots, n\}\).
Each stop \(p\) has a minimal exchange time \(\tau_{ch}(p) \in N\) associated with it.
Often, the exchange time is set to a fixed time \(\tau_{ch}(p) = \tau_{ch}\) for all stops \(p\).
When transferring from a trip \(t\) to another trip \(t' \) within a stop \(p\), the exchange time has to be smaller than the difference in arrival and departure time of the two trips, formally:
\[arrivalTime(t, p) + \tau_{ch}(p) \leq departureTime(t', p) \]
In addition to transfer within stops, RAPTOR also allows footpaths.
Footpaths allow transferring from one stop to another without using public transport. Therefore, they are time-independent.
Each footpath is associated with a travel time \(l(p, p')\).
The input of the RAPTOR algorithm, in addition to the previously described scheduled network, are source stop \(p_s\), and, in the case of a one-to-one query, target stop \(p_t\), as well as the departure time at the source stop \(\tau\).

% Initialization: (l.2-l.8)
RAPTOR operates in rounds.
Before the first round, some variables are initialized.
We denote the earliest possible arrival time at iteration \(i\) with \(\tau_i(p)\) and the best earliest possible arrival time over the course of all iterations with \(\tau^\star(p)\).
For the source stop, \(\tau_p\), we set \(\tau_0(p) =\tau\) and \(\tau^\star(p) = \tau\).
For all other stops, we set \(\tau_0 = \infty\) and \(\tau^\star = \infty\).
In addition, we initialize a set of marked nodes \(M\) only to contain the source stop \(p_s\) and a set of marked route-stop pairs, denoted by \(Q\), to the empty set.
A route-stop pair is simply a tuple that contains a route and one of its stops.
The set of marked stops will contain all stops whose earliest possible arrival time has been updated in the current round.
Similarly, the set of marked route-stop pairs contains the routes of the marked stops, together with the earliest stop of that route that has been marked.


% Summary of steps
Each round consists of three major steps.
In the first step, the routes that have to be iterated are collected.
In the second step, the routes are iterated by "hopping" on their trips.
And in the third stage, potential footpaths are explored.

% Step 1: (l.9 - l.20)
First, we clear the set of marked route-stop pairs \(Q\).
Then, we check the routes that are connected to each marked stop.
For each of these routes, we store the route-stop pair in \(Q\).
However, the routes in \(Q\) should be unique.
If there are two marked stops that are connected to the same route, we choose the stop that is earlier in the sequence of stops of that route.
Now, we clear the set of marked stops.

% Step  2: (l.21-l.33)
We iterate the route-stop pairs in \(Q\) visually depicted in Figure \ref{fig:raptor}.
The following step can be regarded as hopping on the earliest possible trip that we can catch of that route at that stop.
For each route-stop \((r,p)\) pair, we iterate over the stops in \(r\) in the sequence that is associated with \(r\), beginning with \(p\).
We check for the earliest possible trip that we can catch regarding the last arrival time at the current stop \(\tau_{k-1}(p)\) and the minimum exchange time \(\tau_{ch}(p)\).
If there is a trip that is possible to catch, we save it as the current trip \(t_{curr}\) and continue to iterate the stops of the route \(r\).
Now that we are on a trip, we have to check whether we need to update the earliest possible arrival time of the current stop \(\tau_k(p)\) and \(\tau^\star(p)\) by comparing the stop time of the current trip with the best earliest arrival time of that stop \(\tau^\star(p)\), formally:
\[\tau_k(p) = \min\{\tau_k(p), arrivalTime(t_{curr}, p)\}\]
% Here, one optimization comes into play.
%We can also consider the best earliest possible arrival time of the target stop \(\tau^\star(p_t)\).
%Whenever the current stop time is later than that, our current trip won't contribute to finding a faster route to the target stop, and we disregard the potential update.
If an update is necessary, we also add the current stop \(p\) to the marked stops.
Lastly, we check all marked stops for potential footpaths.
Remember: the marked stops are those for which the earliest possible arrival time was updated in this iteration.
For each footpath that is connected to a marked stop, we check whether the earliest possible arrival time of the other stop could be improved by the footpath.
If so, we update the earliest arrival times and mark that stop.
If no stops are marked, there are no new routes to iterate, and the algorithm stops.
After termination \(\tau_k(p)\) contains the earliest possible arrival time at stop \(p\) with at most \(k\) transfers.
\begin{figure}
    \centering
    \includegraphics[width=0.70\textwidth]{Figures/related_work/raptor.pdf}
    \caption{Iterating a Route in RAPTOR}
    \label{fig:raptor}
\end{figure}

% RATPOR algorithm end
One limitation of RAPTOR is the transfer graph, which is used to represent footpaths.
The transfer graph has to be transitively closed, which means that each node has to be connected with an edge to all other nodes that can be reached from that node.
This has the advantage that in the algorithm, we only have to check for direct neighbors of a stop, which is very fast.
In practice, there are many possibilities for how the transfer graph could look.
A realistic transfer graph should be derived from a street network, as passengers should be able to walk from one stop to another using sidewalks.
One could limit the maximum walking distance to keep the transfer graph small.
However, this may remove optimal journeys from the search space.
Creating the transfer graph involves a preprocessing step that becomes exponentially more resource-intensive as the size of the graph increases.
This escalation in computational demand renders the task infeasible quickly as the graph grows larger and more reflective of real-world complexity.

Therefore, finding a fitting transfer graph is challenging.
Through its round-based nature, RAPTOR is able to optimize for two criteria at the same time.
However, RAPTOR cannot incorporate more criteria, and one of the criteria will always be the number of transfers.


\paragraph{McRAPTOR}
\label{subsubsec:mcraptor}

McRAPTOR \shortcite{dellingRoundBasedPublicTransit2015} is an extension of RAPTOR that allows an arbitrary number of criteria.
Like MLC, McRAPTOR also uses the notion of bags containing non-dominating labels.
McRAPTOR does not pose any restrictions on how the objectives are updated during the algorithm.

The algorithm of McRAPTOR only requires slight modifications to the algorithm of RAPTOR.
In the initialization step, each stop \(p\) is assigned an empty bag, except the source stop \(p_s\), which is assigned a bag containing a starting label.
The starting label can be defined as an input but is usually \((\tau, 0, 0, \dots, 0)\), where \(\tau\) is the departure time at the source stop.
When iterating over the route-stop pairs \((r, p)\), McRAPTOR creates a route bag that contains all labels that are in the current bag of \(p\).
In addition, labels in the route bag are associated with a trip.
During the creation of the route bag, each label in the route bag is associated with the first trip that is possible to catch according to the label's earliest arrival time at the current stop \(p\).
Then the route is processed, stop by stop, just like in RAPTOR.
At each stop, the labels in the route bag are updated according to the current trip.
This update must include updating the earliest arrival time but can also include updates to other criteria.
After the route has been processed, the route bag is merged into the bag of the current stop.
Merging a bag \(B_1\) into a bag \(B_2\) means that all labels in \(B_1\) that are not dominated by any label in \(B_2\) are added to \(B_2\) and all labels in \(B_2\) that are dominated by a label in \(B_1\) are removed from \(B_2\).
After the route bag has been merged into the bag of the current stop, the bag of the current stop is merged into the route bag.
Lastly, the trips associated with the labels in the route bag are updated according to the labels' earliest arrival time at the current stop.
Each time a label is added to a stop bag, this stop is marked.
If no stop is marked after a round, the algorithm terminates.
Note that McRAPTOR allows updates to the route bags at any time during processing.
When and how the route bag should be updated depends on the objective and its representation.

While McRAPTOR has a slower runtime than RAPTOR, it is still magnitudes faster than MLC.
However, McRAPTOR still suffers from the same problem as RAPTOR: the transfer graph is hard to compute.


\paragraph{MCR}
\label{subsubsec:mcr}

To overcome the problem of RAPTOR, \shortciteA{dellingComputingMultimodalJourneys2013} introduce Multi-modal Multicriteria RAPTOR (MCR).
MCR modifies McRAPTOR so that the transfer graph must not be transitively closed.
MCR can use the street network as an input directly, so no preprocessing is necessary.
Traversing the street network directly during the algorithm has the benefit of allowing it to update the objectives between trips.
This is important if we want multiple modes of transfer that contain free-floating vehicle-sharing systems.
For example, consider the following case:
For an optimal journey a passenger has to first walk five minutes to a free-floating bicycle, with which the passenger then travels to the next stop.
There is no way to represent this in RAPTOR because the specifics of the transfer depend on the current label, which is unknown before running the algorithm.
Therefore, it is not possible to precompute the transfer graph.

MCR as an algorithm shares substantial similarities with McRAPTOR.
The critical difference in MCR is the substitution of the footpath processing step with the MLC algorithm.
Consequently, MCR can be conceptualized as an algorithm that seamlessly integrates MLC with McRAPTOR, employing each of them in an alternating manner.
The authors find that the bottleneck of MCR is the MLC step.
Therefore, they employ a technique called contraction \shortcite{geisbergerExactRoutingLarge2012} to speed up MLC.
Contraction is a preprocessing technique that reduces the size of the graph by removing nodes and adding shortcut edges.

As previously mentioned, MCR is able to use the street network as the transfer graph and requires no preprocessing.
However, when comparing the runtime of a simple query of MCR and McRAPTOR, MCR is slower, as MLC on the street network is more time-consuming than checking the neighbors of a stop in the transfer graph.
Generally, using MCR or McRAPTOR is a trade-off between runtime and preprocessing time.


\paragraph{ULTRA}
\label{subsubsec:ultra}

\shortciteA{baumUnLimitedTRAnsfersMultiModal2019} propose another algorithm building on MCR, called UnLimited TRAnsfers for Multi-Modal Route Planning (ULTRA).
ULTRA builds on the observation that extensive exploration of the transfer graph, like in MCR, is often unnecessary for transfers between public transport trips but is more crucial for initial and final transfers.
Therefore, they propose a preprocessing step that computes intermediate transfers, contributing to optimal journeys on the transfer graph.
They then use these precomputed transfers as the transfer graph for RAPTOR.
To account for initial and final transfers, ULTRA employs the Bucket Contraction-Hierarchies (Bucket-CH) algorithm \shortcite{geisbergerContractionHierarchiesFaster2008}, an efficient one-to-many approach, together with RAPTOR.

While ULTRA demonstrates a runtime improvement over MCR, it is limited to optimizing only time and the number of transfers.
Using ULTRA in an accessibility analysis setting is also unsuitable because ULTRA runs a reverse Bucket-CH query from the end node to all stops to compute potential final transfers.
This means that ULTRA, unlike MCR, is incompatible with one-to-many queries.

% \paragraph{McTB}
% \label{subsubsec:mctb}

% addresses the problem that ULTRA is not suitable for more than two objectives
% However, it only allows for a particular third objective and not general ones
% \cite{potthoffFastMultimodalJourney2021}

% \paragraph{ULTRA-PHAST}
% \label{subsubsec:ultra-phast}

% addresses the problem that ULTRA is incompatible with one-to-many queries
% However, it does not allow for multiple objectives



\subsection{Public Transport Data}
\label{subsec:public_transport_data}

In practice, a standard data format is needed to retrieve and interpret the data needed to run routing algorithms on public transport networks.
The General Transit Feed Specification (GTFS) \shortcite{mobilitydataGeneralTransitFeed2023} serves as a standardized format for public transportation schedules and associated geographic details.
It is divided into two main components: GTFS Schedule and GTFS Realtime.
GTFS Realtime provides live transit updates.
On the other hand, GTFS Schedule offers information about routes, schedules, fares, and geographic transit details.
For our study and tool, we focus solely on GTFS Schedule and omit considerations related to GTFS Realtime.

Central to the GTFS format are several core concepts.
A route defines the overall path a particular public transport service takes, identified by attributes like name, identifier, and the mode of transport such as bus or subway.
A route is not defined by a single path but rather a collection of possible paths.
For example, in the real world, a route often has two paths, one for each direction.
Similarly, it can happen that a route has to be redirected due to construction work, which would result in a third and fourth path.
A trip refers to a specific run of a vehicle along a route, distinguishing between different timings or sequences of service on the same route.
A stop is a specific point along a route where passengers embark or disembark.
Stops have unique IDs, names, and geographical coordinates.
Stop times specify when a vehicle is expected to be at a particular stop during its trip.
This data pinpoints both the arrival and departure timings at each stop.

% GTFS data is often distributed in plain text (.txt) files, bundled and compressed into a .zip file.
% This packaging makes it compact for distribution and straightforward for developers to parse and use.

In essence, GTFS provides a comprehensive overview of a transit agency's service, covering both the spatial aspects of transit and the temporal aspects.

\subsection{Street Network Data}
\label{subsec:street_network_data}

Just as GTFS provides a standardized format for public transport schedules, the need for a consistent data format for street network information is addressed by OpenStreetMap (OSM) \shortcite{osm-foundationOpenStreetMap2023}.
OSM is a collaborative initiative that offers freely available geographic data.
This data captures various features on the Earth's surface, including roads, regions, points of interest, and more, all of which are effectively represented in a graph structure of nodes and edges (called ways).
In the context of street networks, potentially used by routing algorithms, OSM represents roads and paths using interconnected nodes and ways.
Nodes specify distinct geographical coordinates, defined by latitude and longitude, while ways connect these nodes to define linear structures or area boundaries.
Importantly, these ways have meta-data assigned to them containing information about what vehicles can travel along them and how long they are.
In addition, OSM offers vast amounts of data about points of interest, which are valuable for accessibility analysis.
OSM is extensive, regularly updated, and, most importantly, freely available, which makes it indispensable for projects seeking reproducibility and generalizability.


\subsection{Importance of Multimodality \& Intermodality}
\label{subsec:importance_of_multimodality_and_intermodality}
Various research has underlined the importance of considering multimodality and intermodality in urban planning and transportation.
Recent studies have especially highlighted the synergistic relationship between bicycle sharing and public transport systems, demonstrating their combined potential in improving urban mobility.

\shortciteA{yangImpactPublicBicyclesharing2018} illustrate the significant impact of bicycles in urban transport networks. 
Their results indicate that bicycles notably reduce average transfer times, the average journey length for passengers, and the Gini coefficient, an indicator of network efficiency. 
These findings underscore the role of bicycles in optimizing transit network performance.
Further reinforcing this point, \shortciteA{radzimskiExploringRelationshipBikesharing2021a} identifies a positive correlation between public transport frequency and the number of bicycle trips, particularly for short and medium distances up to 3 km. 
This suggests that the availability of bicycles can complement public transport, particularly for covering the initial or final segments of a journey.
\shortciteA{murphyRoleBicyclesharingCity2015} conducted a survey in Dublin that revealed 39\% of bicycle-sharing users combined this service with another mode of transport, primarily public transport (91.5\%). 
This indicates a high synergy between bicycle sharing and public transport, as commuters frequently use them together.
Similarly, \shortciteA{fishmanBikeShareSynthesis2013} reviewed literature on bicycle sharing and concluded that it is synergistic with public transport.
This synergy is further elucidated by \shortciteA{maBicycleSharingPublic2015}, who, through a linear regression analysis, identified a positive correlation between public transport passenger numbers and bicycle sharing trips. 
They suggest that bicycle sharing effectively addresses the first and last-mile problem, providing a crucial link to and from transit hubs.
\shortciteA{wagnerPublicTransitRouting2017} also contributes to this discussion by demonstrating that unrestricted walking, as part of a multi-modal transit system, significantly reduces travel times compared to limited walking scenarios.
However, it is also noted that computing routes with unrestricted walking is more computationally intensive.

In summary, the interplay between bicycle sharing and public transport is not just complementary but essential for urban planning and transportation.
Furthermore, incorporating unrestricted walking into transit planning, despite its computational challenges, can substantially reduce overall travel times.
Thus, considering both multimodality (the use of multiple modes of transport) and intermodality (the chaining of different modes) is vital when designing urban transportation systems and evaluating accessibility.

\clearpage
\section{Method}
\label{sec:method}

Our method is split into to two parts, the routing algorithm and the accessibility analysis tool.
Our routing algorithm is an applied version of MCR with minor variation to make it more suitable in terms of free-floating vehicle sharing.

% requirements on routing algorithm
% - unrestricted
% - multi-modal (must incorporate scheduled networks (public transfer) and an arbitrary number of other unscheduled networks)
% - multi-objective (must be able to incorporate an arbitrary amount of objectives), whose values update based on the previous values and the current edge (either unscheduled network edge or trip between two stops)
% - inter-modal (the different transport modes may be sequenced in any order (not just bicycles for start and end)

In order to fully grasp the potential of the combination of the sustainable modes of transport, we require our routing algorithm to be \textbf{multi-modal}, \textbf{multi-objective}, and \textbf{unrestricted inter-modal}, and run in a reasonable time.

\textbf{Multi-modal} means that our routing algorithms allows multiple modes of transport, including scheduled transport systems, like public transfer and an arbitrary number of unscheduled transport systems, like walking, cycling and driving.
In addition we require that free-floating vehicle sharing systems are incorporated realistically.
That means, that our routing algorithm must consider that switching to a free-floating vehicle is possible at any location, where a free-floating vehicle is available and parking a free-floating vehicle is possible anywhere where it's allowed.
% note: this extra excludes MCR

\textbf{Multi-objective} means that our algorithm must find all pareto optimal journeys according to an arbitrary amount of objectives.
The algorithm must provide the possibility to update the values of any objective whenever a \textit{movement} occurs.
We define a movement either as an edge traversal in an unscheduled network or a step in the route traversal during McRAPTOR.
In the case of an edge traversal the new objective must be a function of the old objective and the edge weights, formally: \(l' = f(l, w(e))\), where \(l\) and \(l'\) are the old and new labels, respectively, and \(w(e)\) are the weights of the edge that is traversed.
In the case of an update during a step of the route traversal, the new objective must be a function of the old objective (to be continued).

\textbf{Inter-modal} means that the different transport modes may be sequenced in any order.
For example, when considering walking, cycling through a bicycle sharing system and public transport, the algorithm needs to consider journeys with bicycle rides between two consecutive public transport trips.
\textbf{Unrestricted} means that the algorithm fully searches the unscheduled network graphs, and does not pose restrictions like a maximum of 10 minutes walking distance.


% handled:
% dijkstra, mlc, raptor, ultra, mcraptor, mcr
% relate to other algorithms
Both Dijkstra and MLC are not considered due to their impractical runtime.
Furthermore, the need for multi-objective solutions excludes Dijkstra, RAPTOR, and ULTRA.
The requirement for unrestricted inter-modal travel makes RAPTOR and McRAPTOR unsuitable in practical scenarios.
To explain this, let's examine a straightforward example.

Consider the OSM graph of the key regions in Cologne, which comprises 125,176 nodes and 142,074 edges.
For RAPTOR to compute a transitively closed graph, it requires calculating the walking distance between each node.
This computation would yield \(125,176^2 = 15,669,030,976\) edges, a number vastly greater than the original 142,074 edges.

While MCR does support multi-objective solutions with unrestricted inter-modal transfers, it doesn't fully encapsulate the multi-modal concept we require.
Although it theoretically permits various modes of unscheduled transport, it is primarily tailored for station-based vehicle sharing systems.
Our focus, however, is on the increasingly prevalent free-floating systems.
In MCR, unscheduled networks are contracted, leading to the removal of certain nodes.
If an optimal route requires a mode change at a deleted node, MCR will be unable to identify that path.
As a result, MCR is not a viable option for our needs.


In the following section, we detail the modifications made to MCR to tailor it to our requirements.

\subsection{Routing Algorithm}
\label{subs:routing_algorithm}

% I/O
The input of our algorithm is a start time and a start node.
The start node may be any node in the OSM network.
The output of our algorithm are the bags for each node in the OSM network.

% Algorithm
% first phase
Our algorithm is split into two phases, which are repeated iteratively.
The algorithm is depicted in Figure \ref{fig:routing_algorithm}.
In the first phase the walking network is explored through the MLC algorithm.
Walking is not considered a trip and there is no upper limit on the walking distance.
In the initial iteration MLC starts with a single label containing the starting values in the bag of the start node.
All other bags are emtpy.
We run MLC until it converges and retrieve the final bags for each node.

% second phase
In the second phase we explore the public transport network, as well as, all modes of unscheduled travel, except walking.
We do so, because a public transport trip, as well as, a trip with any mode of unscheduled travel, except walking will be counted as a trip.

% second phase - RAPTOR
To explore the public transport network, we run one iteration of McRAPTOR.
To retrieve the proper input bags for McRAPTOR we associate each stop in the public transport network with a node in the unscheduled walking network beforehand.
Then we can use the bags of the nodes in the walking network that are associated with a stop in the public transport network as input for McRAPTOR.

% Second phase - MLC
At the same time, we run MLC again, for each unscheduled mode of transport except walking.
For modes based on free-floating vehicle sharing, we use the bags of nodes in the walking network as an input, where a free-floating vehicle is present.
The output is defined depending on where it is possible to drop off vehicles.
If there are no restrictions the bags of all nodes are used.

% merge
The outputs of the McRAPTOR iteration and all MLC runs are merged.
To do so first the output bags have to be translated into the common nodes of the walking network again.
After that the bags are merged according to the merging rules explained in Section \ref{subsubsec:mcraptor}.

The bags that result the merge are the output of the first iteration and contain all optimal labels after exactly one trip.
To obtain the optimal labels after X trips, both phases have to be repeated X times and the result bags of the second phase in iteration \(i\) are used as the input bags of the first phase in iteration \(i-1\).


\begin{figure}
    \centering
    \includegraphics[scale=0.75]{Figures/method/routing_algorithm}
    \caption{Routing Algorithm}
    \label{fig:routing_algorithm}
\end{figure}

\subsection{Data}
\label{subs:data}

\subsubsection{Data Collection}
\label{subs:data_collection}

Our tool requires several datasets as an input: public transport schedules represented by GTFS files, street networks through OSM files, and data that represents free-floating vehicle sharing.
Notably, both GTFS and OSM data can be accessed publicly and can be easily explored, downloaded and preprocessed with the help of our tool.

For GTFS data, we rely on the Mobility Database \shortcite{MobilityDatabase}.
This database serves as an open-source repository containing links to publicly available GTFS feeds globally, standing as the subsequent version of TransitFeeds \shortcite{OpenMobilityDataPublicTransit}.

To use OSM data in practice various tools and services have been developed.
Among these we use, pyrosm \shortcite{Pyrosm} which is a Python library designed specifically for reading OSM data in different formats and conducting data processing operations.
Through pyrosm, we can automatically fetch data from sources like Geofabrik \cite{GeofabrikDownloadServer} and BBBike \cite{BBBikeExtractsOpenStreetMap}, which are two of the most popular OSM data providers.

Using the combination of these resources, our tool ensures easy access to up-to-date GTFS and OSM data.
This allows for easy reproducibility of our results, as well as, the possibility to use our tool for other cities.

\subsubsection{Data Preperation}
\label{subs:data_preperation}

Our tool is able to trim GTFS data to a specific bounding box.
This is especially useful for country-size GTFS feeds.

The GTFS data is also cleaned and converted into a format that is more suitable for RAPTOR.

Specifically, there are two major incompatibilities between the GTFS specification and RAPTOR's notion of routes and trips.
Firstly, each trip belonging to a single route in RAPTOR visits the same stops in the same order.
It is not possible that a trip skips some stops that another trip of the same route visits, much less use a completely different sequence of routes.
In GTFS routes do allow that, as they are much more a group of trips that is presented to the rider under the same name or identifier.
Secondly, GTFS trips allow visiting the same stop multiple times, which is not allowed in RAPTOR.

To overcome these difference our tool splits up routes into smaller routes, that follow the same sequence of stops.
Additionally, it also removes circular trips, altogether.

Similarly, our tool is also able to extract an actual graph from the OSM network.
To do so it utilizes pyrosm.
After extracting the graph from the OSM network, the graph is trimmed to the convex hull of the GTFS stop extended by a small buffer zone.
As a last cleaning step, we remove all nodes, that are not part of the largest weakly connected component.
A weakly connected component is a subgraph in which, if all directed edges were treated as undirected, any two vertices from the subgraph would be connected.
Multiple weakly connected components in graphs derived from OSM data, mostly happen at the border of the considered area and can be neglected.

\subsection{Accessibility Analysis}
\label{subs:accessibility_analysis}

To evaluate the accessibility in cities, we employ a metric that is an implementation of the 15-minute city concept. The concept measures how fast the access to a variety of important amenities is.
To measure this, we categorize amenities into seven essential services: grocery, education, health, banks, parks, sustenance, and shops. (as described by)
Each category is populated with Points of Interest (POIs) sourced from OSM, providing a comprehensive database of locations.

Each service category encapsulates several POIs. For instance, the "Parks" category may include multiple locations tagged in OSM as "leisure: park" or "leisure: dog park".

The core of our metric is the determination of temporal proximity to these amenities. 
For each category, we calculate the minimum travel time required to reach at least one POI of that category. 
The metric is then defined as the maximum value among these minimal times across all categories. 
This approach yields a singular measure that reflects the most significant time distance barrier within an urban area, which effectively captures the least accessible essential service category for any given area.

This metric is critical in assessing the performance of a neighborhood or a city at large against the 15-minute city ideal. It is not an average of accessibility across services but rather highlights the area of greatest need, providing a clear target for urban development and improvement.

By leveraging this metric, we aim to help city planners to create urban environments that prioritize sustainability, enhance the well-being of residents, and reduce dependency on vehicular transport, thus contributing to the broader goals of efficient urban planning and improved quality of urban life.

% explain different modes - this belongs to related work
Traditionally, the 15-minute city concept is applied to walking and or cycling and ignores other modes of transport.
Some researchers, in the context of location-based metrics, even go as far to only calculate the bee-line distance to the nearest amenity and ignore the street network altogether \cite{gastnerOptimalDesignSpatial2006}.

We, however, believe that to accurately determine the accessibility of a city, all modes of transport must be considered, and the routing needs to be as realistic as possible.
We will therefore calculate our metric for various combinations of modes of transport, namely driving with a personal car+walking, free-floating bicycle sharing+walking, public transport+walking, free-floating bicycle sharing+public transport+walking, and walking.
The car mode will serve as a baseline metric and show how competitive more sustainable modes of transport are.


\clearpage
\section{Experiment}
\label{sec:experiment}

We apply our method to the city of Cologne to retrieve insights about how different modes of transport interact and how they contribute to the city's accessibility.
As we want to compare different modes of transport, we will calculate the metric multiple times for different scenarios, each with a different combination transport modes.

\subsection{Scenarios}
\label{subs:scenarios}

No matter what mode of transport we choose, we always allow for walking for two reasons.
First, walking is the most accessible mode of transport, available to almost everyone and without additional costs.
Second, most other modes of transport require walking at some point, be it to the next bus stop or the next available bicycle.

The first scenario we consider is our baseline, which only considers walking.
This scenario measures what is possible without any additional infrastructure. 
Distinct from other scenarios, it does not require any cost, thus presenting the most basic form of urban mobility.

Building on this, the second scenario we consider is using a personal car.
We consider this scenario the benchmark scenario, as we hope to achieve similar (or even better) results with more sustainable modes of transport.
Therefore, we use it to answer the question of how competitive sustainable modes of transport are compared to the traditional mode of travel by car. 

Transitioning from the benchmark scenario, the third scenario focuses on public transport. 
It is essential to understand the effectiveness and accessibility of urban transit systems. 
This scenario evaluates how well-connected and time-efficient public transportation networks are and their role in reducing reliance on personal vehicles. 
It also investigates the impact of public transport on urban mobility and its potential to contribute to a more sustainable urban environment. 
Specifically, it assesses whether public transport is a viable alternative to the personal car and whether it offers significant advantages over walking, considering the X-minute city metric.
In contrast to the previous scenarios the public transport scenario also incorporates uncertainty.
As the public transport network is scheduled, i.e. time-dependent, we have to consider multiple different starting times, so that we get holistic assessment of this mode.

Next, in the fourth scenario, we focus on the dynamics of bicycle-sharing systems. 
This scenario is essential for assessing the feasibility and attractiveness of cycling as a primary mode of transportation in urban areas. 
We will directly compare it to the public transport scenario to understand which sustainable mode of transport is superior.
Similarly to the public transport scenario, we also incorporate uncertainty in this scenario.
The availability of bicycles fluctuates temporally and spatially.
Therefore, we consider multiple different bicycle availability configurations.

Finally, the fifth scenario combines public transport and bicycle sharing, offering insights into the synergy between these two modes of transport.
For brevity, we refer to this scenario as the combined scenario.
This integrated approach mirrors a growing trend in urban mobility solutions, where multi-modal transport options are increasingly favored. 
It underscores how this combination can bridge the gaps in accessibility and efficiency found when each mode is used independently. 
This scenario is expected to be the most competitive against cars, offering a comprehensive and sustainable urban transit model that could reshape the landscape of city mobility.

We summarize the scenarios in Table \ref{table:scenarios}.
The specific configuration of the module matrices for each scenario can be found in Appendix \ref{app:experiment_module_matrix_configuration}.

\begin{table}[h]
\centering
\begin{tabular}{|p{4cm}|p{5cm}|p{4cm}|}
\hline
\textbf{Scenario} & \textbf{Modules} & \textbf{Uncertainty} \\
\hline
Walking (baseline) & Walking & - \\
\hline
Personal car (benchmark) & Personal vehicle, walking & - \\
\hline
Public transport & Public transport, walking & Start time \\
\hline
Bicycle sharing & Vehicle sharing, walking & Bicycle availability \\
\hline
Combined & Public transport, bicycle sharing, walking & Start time \& bicycle availability \\
\hline
\end{tabular}
\caption{Scenarios for Urban Mobility Analysis}
\label{table:scenarios}
\end{table}


\subsection{Data}
\label{subs:data}

We use four different datasets to calculate the Pareto sets of our metric for the different scenarios.
First, we require data that depicts the street network of the city of Cologne.
Second, we need to know the locations of the POIs we want to reach.
Third, we need to know the locations of the public transport stops and the schedules of the public transport.
For the bicycle-sharing scenario, we also need to know the locations of the bicycles.
Lastly, we also use land use data to identify where residential areas are located to calculate our metric only for these areas.

As we will query spatial datasets of various formats from different sources, the area covered by the datasets will not be the same.
Therefore, we first define an area of interest and trim the datasets to this area.
In our case, this area is defined as the area of the administrative district of Cologne, specifically the "Stadtkreis Köln".
We retrieve the specific boundary of this area with the help of the Overpass API \shortcite{osm-foundationOverpassAPI2023}.
The specific query can be found in Appendix \ref{app:overpass_query}, and the resulting region can be seen in Figure \ref{fig:area_plus_buffer}.

\begin{figure}
  \begin{center}
    \includegraphics[width=0.55\textwidth]{Figures/experiment/area_plus_buffer.png}
  \end{center}
  \caption{Area of Interest and Buffer Region}\label{fig:area_plus_buffer}
\end{figure}


The Figure additionally depicts a buffer zone surrounding the area of interest.
This expanded area incorporates an additional buffer of approximately 5 km, which is essential as it directly impacts the plausibility of our accessibility analysis.
Specifically, employing a larger underlying street network than the core area of interest circumvents the problem of underestimating accessibility in border regions.

\subsubsection{Street Network \& POIs}
\label{subs:street_network_pois}

For the street network and the POIs, we use data from OpenStreetMap (OSM) \shortcite{osm-foundationOpenStreetMap2023}.
Various tools and services have been developed to use OSM data in practice.
Among these, we use Pyrosm \shortcite{tenkanenPyrosm2023}, a Python library designed specifically for reading OSM data in different formats and conducting data processing operations.
With the help of Pyrosm, we can automatically fetch data from sources like Geofabrik \shortcite{geofabrikGeofabrikDownloadServer2018} and BBBike \cite{schneiderBBBikeExtractsOpenStreetMap2023}, which are two of the most popular OSM data providers.
In our case, we use the data for the city of Cologne from BBBike.
However, due to the flexibility of Pyrosm, it is easily possible to use data from other sources and expand our analysis to other cities.

After retrieving the data, we retrieve a graph representation of the street network trimmed to the buffered area of interest.
Using the buffered region is important because without it, calculating our metric at the border of the area of interest would result in a higher value than the actual value.
As a last cleaning step, we remove all nodes not part of the largest weakly connected component.
A weakly connected component is a subgraph in which any two vertices from the subgraph would be connected if all directed edges were treated as undirected.
Multiple weakly connected components in graphs derived from OSM data, mostly happen at the border of the considered area and can be neglected.

Because we consider multiple different modes of transport on the network, it is essential to filter out all edges that are not accessible by the respective mode of transport.
To do so, we use Pyrosm's built-in filtering functionality.
For reproducibility, we list the filters that Pyrosm uses in Appendix \ref{app:pyrosm_network_filter}.

To retrieve the POIs, we use the Overpass API \shortcite{osm-foundationOverpassAPI2023}.
We retrieve all POIs that fall into one of our predefined categories specified in Section \ref{subsec:metric} inside the area of interest plus the buffer mentioned before.

\subsubsection{Public Transport}
\label{subs:public_transport}

We use the General Transit Feed Specification (GTFS) \shortcite{mobilitydataGeneralTransitFeed2023} to handle public transport data.
We rely on the Mobility Database \shortcite{mobilitydataMobilityDatabase2023} to retrieve it.
This database serves as an open-source repository containing links to publicly available GTFS feeds globally, standing as the subsequent version of TransitFeeds \shortcite{mobilitydataOpenMobilityDataPublicTransit2023}.
Similarly to the OSM data, we trim the GTFS data to the area of interest plus the buffer.
The GTFS data is also cleaned and converted into a format more suitable for our algorithm, specifically McRAPTOR, which is part of our algorithm.
Specifically, two significant incompatibilities exist between the GTFS specification and RAPTOR's notion of routes and trips.
Firstly, each trip belonging to a single route in RAPTOR visits the same stops in the same order.
It is impossible for a trip to skip some stops that another trip of the same route visits, much less use a completely different sequence of routes.
In GTFS, routes allow that, as they are much more a group of trips presented to the passenger under the same name or identifier.
Secondly, GTFS trips allow visiting the same stop multiple times, which is not permitted in RAPTOR.
To overcome these differences, we split up routes into smaller routes following the same stop sequence.
Additionally, we also remove circular trips, altogether, as they only in a few cases in Cologne and are not relevant for our analysis.


\subsubsection{Bicycle Sharing}
\label{subs:bicycle_sharing}

Our bicycle-sharing data was retrieved by continuously polling the NextBike API over the course of one year.
The data consists of all trips made with the NextBike system in Cologne from the 15th of January 2022 to the 31st of August 2023.
To get representative samples of the locations of all bicycles, we employ the following strategy.
We first discretize the data spatially and temporally.
For the temporal discretization, we derive the location of each bicycle every hour.
For the spatial discretization, we use H3 hexagons with a resolution of 9.
The resulting data shows how many bicycles were at each hexagon at each hour.
This data is then used as an input for k-medoids clustering \shortcite{rdusseeun1987clustering} with a k of 4.
K-medoids, also known as PAM (Partitioning Around Medoids) algorithm, is a clustering technique that partitions a dataset into K clusters, where each is assigned a medoid, the most centrally located object in a cluster. 
Unlike K-means, which uses mean values as cluster centers, K-medoids use an actual data point as the center of a cluster.
This has the advantage that the centers are part of the dataset and, therefore, are realistic samples.
We use the resulting medoids as bicycle availability configurations.

\subsubsection{Land Use}
\label{subs:land_use}

We use the land use data from the CORINE Land Cover (CLC) project \shortcite{european-environment-agencyCORINELandCover2010} to identify the residential areas.
The data covers the whole of Europe and is publicly available, making it possible to expand our analysis to other cities in Europe.
We trimmed the data to the area of interest and then filtered for the land use types "Continuous Urban Fabric" and "Discontinuous Urban Fabric".
These two land use types represent the residential areas of the city.
The residential areas inside the area of interest are shown in Figure \ref{fig:input_hexagons_residential_areas}.
Additionally, the Figure shows the hexagons of resolution nine found inside the residential areas.

\begin{figure}
  \begin{center}
    \includegraphics[width=0.65\textwidth]{Figures/experiment/input_hexagons_residential_areas.png}
  \end{center}
  \caption{Area of Interest and Buffer Region with Residential Areas and Input Hexagons}
  \label{fig:input_hexagons_residential_areas}
\end{figure}


\subsection{Assumptions}
\label{subs:assumptions}

To calculate our metric, we have to abstract from reality to some degree.
We do so by making the following plausible assumptions.

Firstly, we assume that traveling along an edge of the street network by walking, cycling, or driving is always proportional to the length of the edge.
To obtain the time it takes to travel along an edge, we divide the length of the edge by the speed of the mode of transport.
The different speeds for the different modes of transport are listed in Table \ref{table:speeds}.

\begin{table}[h]
\centering
\begin{tabular}{|c|l|}
\hline
\textbf{Mode} & \textbf{Speed (m/s)} \\
\hline
Walking & 1.4 \\
\hline
Cycling & 4.0 \\
\hline
Driving & 11.0 \\
\hline
\end{tabular}
\caption{Speeds for Different Modes of Transport}
\label{table:speeds}
\end{table}

The walking speed is consistent with the measurement that \shortciteA{willberg15minuteCityAll2023} made in their study.

We also pose some assumptions on the transitioning between different modes of transport and, in the case of public transport, the transfer time at the stops.
We assume a fixed time of one minute for the transfer time at stops.
To transition from any OSM network-based mode of transport to public transport, we assume that the stop is precisely at the location of the closest node of the OSM network.
As OSM networks contain public transport stops, there should be no difference between the two.
Similarly, we assume that the bicycles are located at the closest node of the OSM network.
This assumption is reasonable because the OSM network, especially in the city, is very dense.
We also assume that bicycles and cars can be parked anywhere on their network for simplicity.


% \subsubsection{Pricing}
% \label{subs:pricing}
\paragraph{Pricing}

We implement a pricing scheme in our scenarios that represents the real-world circumstances as closely as possible.

For bicycle sharing, we use the pricing scheme of NextBike, which is \euro{1} every 15 minutes.
To depict this, we add a hidden value to the labels processed in MLC that shows how long the current bicycle trip is.
As two consecutive bicycle trips are considered separately, we nullify this hidden value after each run of MLC.

For public transport, we use the pricing scheme of the Cologne Transport Authority (KVB), which is \euro{2.20} for trips that span four stops or less.
For any trip that spans more than four stops or any multitude of trips, the KVB charges \euro{3.20}.
To depict this, we add hidden values to the labels processed in McRAPTOR, representing how many stops the traveler has already traversed.
Because two consecutive trips are considered together, we don't need to nullify the hidden value.

For personal vehicles, we use a cost per minute of 19 cents, which we derived from the average cost per kilometer of 28 cents given by \cite{kieferSpritkostenrechner2023} and an average speed of \SI{40}{\kilo\meter\per\hour}.
These costs incorporate fuel, repair, insurance, and tax, but not acquisition costs.


\clearpage
\section{Results}
\label{sec:results}

This section presents the findings of our comprehensive analysis, addressing the research questions outlined in the introduction.
Recall, our primary inquiries revolved around understanding the role of bicycle sharing and public transport in shaping cities into 15-minute cities (\ref{rq:bicycle_pt}), the impact of cost on accessibility (\ref{rq:cost_accessibility}), the measurement of accessibility considering multiple transport modes and associated costs (\ref{rq:measure_accessibility}), and deriving specific urban planning recommendations for Cologne (\ref{rq:recommendations}).

The results are the output of our experiment, which consists of a novel method for accessibility-based planning that incorporates multiple modes of transport, extending the 15-minute city concept to include a broader range of sustainable transport options while also considering cost.
The data retrieved from our experiments consists of two parts:
Firstly, for each (sub)-scenario and hexagon, we get the Pareto set of the X-minute city metric and cost.
Secondly, we also retrieve the more fine granular version where we get a Pareto set of the time it takes to get to the closest POI for a specific category.
In the following subsections, we delve into this data and observe our method's runtime and memory usage.



\subsection{Runtime Observations}
\label{subsec:runtime_observations}

% TODO: data prep is missing
% TODO fill in actual numbers
Observing the runtime and memory usage required to run our experiment enables us to evaluate the practicality of our approach.
To execute our experiment, we used a machine with an AMD Ryzen 7 5800H CPU and 32 GB of RAM.
As explained in Section \ref{sec:method} our routine is split into three parts, the input, main, and metric routine.
The input routine took X minutes to run and required no more than Y gigabytes of memory.
The main routine took the longest with X minutes and required Y gigabytes of memory, however, we only utilized eight of the 16 available cores for parallelization.
Lastly, the metric routine took X minutes to run and required Y gigabytes of memory.

We found that the memory consumption primarily stems from the graph-based representation of the street network of Cologne.

\subsection{15-Minute City Metric}
\label{subsec:15_minute_city_metric}

% table (mean, quantiles) of the optimal X-minute city metric
Table \ref{tab:optimal_x_minute_city_metric} shows the mean, as well as, the 25\%, 50\%, and 75\% quantiles of the optimal X-minute city disregarding the cost for each scenario over all hexagons.

\begin{table}
  \caption{Optimal X-minute City Metric Over All Hexagons Disregarding Cost}
  \label{tab:optimal_x_minute_city_metric}
  \begin{center}
    \begin{tabular}{lrrrr}
       & mean & 25\% & 50\% & 75\% \\
      scenario &  &  &  &  \\
      bicycle & 12.45 & 7.25 & 10.75 & 15.50 \\
      bicycle\_public\_transport & 11.51 & 7.25 & 10.33 & 14.31 \\
      car & 3.21 & 2.00 & 3.00 & 4.00 \\
      public\_transport & 12.78 & 9.00 & 12.00 & 16.00 \\
      walking & 14.09 & 9.00 & 12.00 & 17.00 \\
    \end{tabular}
  \end{center}
\end{table}

Our findings indicate that cars enable the fastest access to all necessary Points of Interest (POIs), with an average accessibility time of 3.21 minutes. 
This mode of transport significantly outpaces other methods, establishing a benchmark for urban mobility efficiency.
However, remember that our car scenario is very optimistic and these numbers should be taken cautiously.

In contrast, sustainable modes of transport, such as bicycles, public transport, a combination of bicycles and public transport, and walking, demonstrate similar accessibility times. 
These modes record average times ranging from 11.5 to 14 minutes, with walking being the least time-efficient mode at an average of 14.09 minutes. 

Integrating bicycles with public transport emerges as the most time-efficient sustainable mode, with an average time of 11.51 minutes. 
A direct comparison between public transport and walking shows that the time savings offered by public transport stand at 1 minute and 28 seconds. 
However, this benefit is not evenly distributed across all areas.
The analysis of quantiles reveals that the time improvement only establishes at the 75\% quantile with a 2-minute gain, while the 25\% and 50\% quantiles don't show any improvements.

Similarly, adding public transport to bicycle sharing improves the average optimal time to reach all categories by 43 seconds.
Again, this improvement is not evenly distributed but only applies to the 25\% worst hexagons.
Specifically, we see no improvement from bicycle sharing to public transport in the 25\% and 50\% quantiles, but a 1-minute improvement in the 75\% quantile.
While there is an improvement in the mean and 75\% quantile, it is not as large as the improvement from walking to public transport.

We can make the same observation from the standpoint of adding bicycle sharing to walking and public transport.
Adding bicycle sharing to public transport, the data indicates an improvement in the average accessibility time, reducing it by 1 minute and 16 seconds.
In contrast to adding public transport to bicycles, this improvement already occurs for the 25\% quantile and is, therefore, more evenly distributed across all hexagons.
Adding bicycles to the walking scenario presents an average time reduction of 1 minute and 28 seconds, which denotes a significant enhancement in the accessibility metric. 
Again, this improvement already occurs at the 25\% quantile, showing that the improvements gained through bicycle sharing are more evenly distributed across all hexagons.

% visualization of the distribution of the optimal X-minute city metric
We can observe a similar pattern when visualizing the distribution of the optimal X-minute city metric in Figure \ref{fig:optimal_x_minute_city_metric}.
\begin{figure}
  \begin{center}
    \includegraphics[width=0.65\textwidth]{Figures/results/minute_city_metric/best_x_minute_city}
  \end{center}
  \caption{Distribution of Optimal X-Minute City Metric}
  \label{fig:optimal_x_minute_city_metric}
\end{figure}
As we can see initially (for the most accessible hexagons), public transport and walking are the same, but as we move to less accessible hexagons, public transport improves.
In addition, the public transport scenario is worse than the pure bicycle sharing scenario but can catch up and even overtake it as we move to less accessible hexagons.
The same pattern can be observed when comparing the bicycle-sharing scenario to the combined scenario of bicycle-sharing and public transport.
Initially, the combined scenario is the same as the bicycle-sharing scenario, but as we move to less accessible hexagons the combined scenario becomes better.
Similarly, when comparing the bicycle-sharing scenario to the combined scenario, we see that the combined scenario provides much better accessibility initially, but as we move to the least accessible hexagons, both become the same.
Generally, adding public transport can flatten the drastic increase of the optimal X-minute city metric at the end of the distribution.


Figure \ref{fig:optimal_map} shows the optimal X-minute city metric for each hexagon over all sustainable modes of travel, i.e., excluding the car scenario.
\begin{figure}
  \begin{center}
    \includegraphics[width=0.45\textwidth]{Figures/results/minute_city_metric/optimal_map}
  \end{center}
  \caption{Map of Optimal X-Minute City Metric}
  \label{fig:optimal_map}
\end{figure}
We can see that the least accessible hexagons require 44 minutes to reach all categories if only sustainable modes of travel are used.
The least accessible regions are suburban areas in the north and south of Cologne. 
The region on the left bank of the Rhine River next to Leverkusen, which is the district of Merkenenich, is especially inaccessible.

Figure \ref{fig:optimal_map_per_scenario} shows multiple maps of the optimal X-minute city metric for each hexagon, one for bicycle sharing, one for public transport, and one for walking.
\begin{figure}
     \centering
     \begin{subfigure}[b]{0.3\textwidth}
         \centering
         \includegraphics[width=\textwidth]{Figures/results/minute_city_metric/public_transport_optimal_map}
         \caption{Public Transport}
         \label{fig:public_transport_optimal_map}
     \end{subfigure}
     \hfill
     \begin{subfigure}[b]{0.3\textwidth}
         \centering
         \includegraphics[width=\textwidth]{Figures/results/minute_city_metric/bicycle_optimal_map}
         \caption{Bicycle Sharing}
         \label{fig:bicycle_optimal_map}
     \end{subfigure}
     \hfill
     \begin{subfigure}[b]{0.3\textwidth}
         \centering
         \includegraphics[width=\textwidth]{Figures/results/minute_city_metric/walking_optimal_map}
         \caption{Walking}
         \label{fig:walking_optimal_map}
     \end{subfigure}
        \caption{Map of Optimal X-Minute City Metric per Scenario}
        \label{fig:optimal_map_per_scenario}
\end{figure}
The areas in and around the city center are more accessible by bicycle sharing than by public transport and walking.
In the east of the city, near the forest "Königsforst", we see the district of Rath/Neumar, with low accessibility for all scenarios.
However, one can see that the region is more accessible by public transport than by bicycle sharing and walking.

\subsection{Cost of 15-Minute City}
\label{subsec:cost_of_15_minute_city}

% table (mean, quantiles) of required cost
Table \ref{tab:required_cost} shows the mean, the 25\%, 50\%, and 75\% quantiles and the maximum of the costs that are required to achieve the optimal value for the X-minute city shown in Section \ref{subsec:15_minute_city_metric}.
\begin{table}
  \caption{Required Cost for Optimal Over All Hexagons}
  \label{tab:required_cost}
  \begin{center}
    \begin{tabular}{lrrrrr}
     & mean & 25\% & 50\% & 75\% & max \\
    scenario &  &  &  &  &  \\
    bicycle & 0.39 & 0.00 & 0.50 & 0.75 & 1.00 \\
    bicycle public transport & 0.87 & 0.00 & 0.75 & 1.30 & 3.95 \\
    car & 0.37 & 0.19 & 0.38 & 0.38 & 1.33 \\
    public transport & 0.65 & 0.00 & 0.00 & 1.47 & 3.20 \\
    walking & 0.00 & 0.00 & 0.00 & 0.00 & 0.00 \\
    \end{tabular}
  \end{center}
\end{table}
We can immediately see that there is no cost for hexagons at the 25\% and 50\% quantile when using public transport, implying that public transport is not used at all for those hexagons.
Looking at the 75\% quantile and the maximum required cost for an optimal x-minute city metric for public transport, we see that the benefits we observed earlier come at a cost.
Similarly, bicycle sharing and the combined mode have zero cost at the 25\% quantile, implying that they are not used for those hexagons.

Looking at the maximum cost values of each sustainable mode of transport shows us that bicycles never require more than \euro{1}.
As \euro{1} allows to travel a maximum of 15 minutes, it is never necessary to travel longer than 15 minutes after a bicycle is reached.
The maximum cost incurred by public transport is \euro{3.20}, the long-distance ticket of public transport (more than four stops) is used.
Similarly, in the combined scenario, the long-distance ticket is used with a 15-minute ride of bicycle sharing in at least one sub-scenario, resulting in the maximum price of \euro{3.95}.

We can make similar observations with more granularity when looking at the distribution of the required cost in Figure \ref{fig:maximum_required_cost_for_x_minute_city}.
\begin{figure}
  \begin{center}
    \includegraphics[width=0.65\textwidth]{Figures/results/cost/maximum_required_cost_for_x_minute_city}
  \end{center}
  \caption{Maximum Required Cost for Optimal X-Minute City Metric}
  \label{fig:maximum_required_cost_for_x_minute_city}
\end{figure}
A new pattern stands out when comparing public transport and the combined mode.
We see that the combined mode has higher costs earlier, surpassed by public transport, only to be surpassed again by the combined mode.
% discussion stuff (not sure where to put this)
The first price increase in the combined mode can be explained by the \euro{1} cost of 15-minute bicycle sharing.
Then public transport surpasses the combined mode. 
This probably is because bicycle sharing is more cost-efficient and can compensate for public transport.
The fact that the combined mode then surpasses public transport again most likely stems from the fact that the combined mode can achieve faster access than public transport by using bicycle sharing, which is more expensive than a short-distance ticket alone.
% END

Figure \ref{fig:cost_map_per_scenario} shows the cost required to reach the optimal X-minute city metric for each hexagon for public transport, bicycle sharing, and the combined scenario of bicycle sharing and public transport.
Note that we don't show the cost for the walking scenario, as it is always \euro{0}.
In these figures, we see that sometimes the cost is zero.
As the portrayed scenarios all have costs associated with them, a cost of zero means that only walking is used.
\begin{figure}
     \centering
     \begin{subfigure}[b]{0.3\textwidth}
         \centering
         \includegraphics[width=\textwidth]{Figures/results/cost/public_transport_cost_map}
         \caption{Public Transport}
         \label{fig:public_transport_cost_map}
     \end{subfigure}
     \hfill
     \begin{subfigure}[b]{0.3\textwidth}
         \centering
         \includegraphics[width=\textwidth]{Figures/results/cost/bicycle_cost_map}
         \caption{Bicycle Sharing}
         \label{fig:bicycle_cost_map}
     \end{subfigure}
     \hfill
     \begin{subfigure}[b]{0.3\textwidth}
         \centering
         \includegraphics[width=\textwidth]{Figures/results/cost/bicycle_public_transport_cost_map}
         \caption{PT + Bicycle}
         \label{fig:bicycle_public_transport_cost_map}
     \end{subfigure}
       \caption{Map of Required Cost for Optimal for Each Hexagon}
        \label{fig:cost_map_per_scenario}
\end{figure}
We see almost in all hexagons in and around the city center, where NextBike's flex zone is located, the cost for the bicycle sharing scenario is \euro{1}.
This sometimes also extends outside the city center.

The cost of public transport is more scattered around the whole region. 
We can mostly see single hexagons in the city's center and small groups of hexagons outside the city that have costs have than zero.

\subsection{Interaction Between Cost and 15-Minute City Metric}
\label{subsec:interaction_between_cost_and_15_minute_city_metric}

Next, we will examine the interaction between the cost and the optimal X-minute city metric.
To do so, we will investigate the mean Pareto front of the X-minute city metric and cost over all hexagons.
To understand this graph, we first examine the Pareto front of a single hexagon.

Figure \ref{fig:example_pareto_front} shows the Pareto front for an example hexagon.
The x-axis shows the cost, and the y-axis shows the X-minute city metric.
The line shows us what X-minute city metric is achievable for a given cost in a specific scenario.

\begin{figure}
  \begin{center}
     \includegraphics[width=0.5\textwidth]{Figures/results/metric_cost/example_profile}
  \end{center}
  \caption{Example Pareto Front}
  \label{fig:example_pareto_front}
\end{figure}

In our example, all modes begin with being able to reach all categories within 22 minutes for a cost of \euro{0}.
Increasing the cost only yields improvements when reaching a cost of \euro{1}, where the bicycle and combined scenarios can reach all categories within approximately 10 minutes.
Further, increasing the price to \euro{2.20} improves the public transport scenario, where reaching all categories within approximately 19 minutes is now possible.
Further cost increases do not yield any improvements for any scenario.

We can also quantify the value of the improvements as seen in Table \ref{tab:differences_in_example_hexagon}.
This table shows all the steps with their cost position and magnitude, visible in the previous graph.
In addition, we can calculate the benefit in minutes per one euro of cost to make their value more comparable.
\begin{table}
  \caption{Steps in Example hexagon}
  \label{tab:differences_in_example_hexagon}
  \begin{center}
    \begin{tabular}{lrrrl}
     improvement & at cost & minute per euro & scenario \\
     11.75 & 1.00 & 11.75 & Bicycle \\
     11.75 & 1.00 & 11.75 & Combined \\
     3.00 & 2.20 & 1.36 & Public Transport \\
    \end{tabular}
  \end{center}
\end{table}
As we can see, the bicycle scenarios' increase at a cost of \euro{1} is larger than the public transport scenario's increase and has a higher value per euro.

% generalization
To generalize these findings over all hexagons, we take the average over the X-minute city for each cost and scenario to generate an average Pareto front.
The resulting Pareto front can be seen in Figure \ref{fig:mean_time_per_cost}.

\begin{figure}
  \begin{center}
     \includegraphics[width=0.5\textwidth]{Figures/results/metric_cost/mean_time_per_cost}
  \end{center}
  \caption{Mean Time per Cost for All Scenarios}
  \label{fig:mean_time_per_cost}
\end{figure}

Similarly to the example of the single hexagon from before, we can see improvements for the bicycle scenario and the combined scenario at the cost of \euro{1} of about 1.5 minutes.
We can also see the improvements in public transport at a cost of \euro{2.20}.
Unlike the example of the single hexagon, we can also see the improvement at the cost of \euro{2.20} for the combined scenario.
Lastly, there is a slight improvement for the public transport scenario and the combined scenario at a cost of \euro{3.20}.

To compare these improvements, we can again look at the differences in Table \ref{tab:differences_in_mean_pareto_front}.
We won't analyze the differences in the combined scenario, as prior improvements of other modes may skew them and are therefore hard to interpret.

\begin{table}
  \caption{Steps in Mean Pareto Front}
  \label{tab:differences_in_mean_pareto_front}
  \begin{center}
    \begin{tabular}{lrrrrl}
     & improvement & at cost & cost diff & minute per euro & scenario \\
     & 1.684 & 1.000 & 1.000 & 1.684 & bicycle \\
     & 1.282 & 2.200 & 2.200 & 0.583 & public transport \\
     & 0.074 & 3.200 & 1.000 & 0.074 & public transport \\
    \end{tabular}
  \end{center}
\end{table}

We see that the improvements of the bicycle scenarios at the cost of \euro{1} are the largest, with an improvement of 1.68 minutes, and also the most cost-effective, with a value of 1.68 minutes per euro.
They are followed by the improvements of the public transport scenario at a cost of \euro{2.20} with an improvement of 1.28 minutes and a value of 0.58 minutes per euro.
The improvement at a cost of \euro{3.20} is minimal and the least cost-effective.

Next, we are going to look at the quantiles of the aggregated Pareto front.
Figure \ref{fig:quantile_time_per_cost} shows the 25\%, 75\%, and 90\% quantiles of the aggregated Pareto front.
The 25\% quantile gives us insights about the more accessible areas in the city.
Note that because we aggregate all the values of the X-minute city metric for a single cost and scenario at a time, the 25\% quantile Pareto front does not necessarily reflect the same 25\% of hexagons for each cost.

\begin{figure}
     \centering
     \begin{subfigure}[b]{0.48\textwidth}
         \centering
         \includegraphics[width=\textwidth]{Figures/results/metric_cost/quantile_25_time_per_cost_for_each_scenario_without_car.png}
         \caption{25\% quantile time per cost for all scenarios}
         \label{fig:25_quantile_time_per_cost}
     \end{subfigure}
     \hfill
     \begin{subfigure}[b]{0.48\textwidth}
         \centering
         \includegraphics[width=\textwidth]{Figures/results/metric_cost/quantile_75_time_per_cost_for_each_scenario_without_car.png}
         \caption{75\% quantile time per cost for all scenarios}
         \label{fig:75_quantile_time_per_cost}
     \end{subfigure}
     \hfill
     \begin{subfigure}[b]{0.48\textwidth}
         \centering
         \includegraphics[width=\textwidth]{Figures/results/metric_cost/quantile_90_time_per_cost_for_each_scenario_without_car.png}
         \caption{90\% quantile time per cost for all scenarios}
         \label{fig:90_quantile_time_per_cost}
     \end{subfigure}
        \caption{Map of Optimal X-minute City Metric per Scenario}
        \label{fig:quantile_time_per_cost}
\end{figure}

The 25\% quantile Pareto front shown in Figure \ref{fig:25_quantile_time_per_cost} only contains a single improvement at the cost of \euro{1} for scenarios containing bicycle sharing of 1.75 minutes with a cost-effectiveness of 1.75 minutes per euro.


The 75\% quantile Pareto front shown in Figure \ref{fig:75_quantile_time_per_cost} with its steps shown in Table \ref{tab:differences_in_75_quantile_pareto_front} also has a similar improvement of 1.5 minutes at the cost of \euro{1} for bicycle scenarios.
In addition to that, it also shows a smaller increase at \euro{2.20} for public transport scenarios of 1 minute and an even smaller increase at 3.20 euros for bicycle sharing and public transport scenarios.

\begin{table}
  \caption{Steps in 75\% Quantile Pareto Front}
  \label{tab:differences_in_75_quantile_pareto_front}
  \begin{center}
    \begin{tabular}{lrrrrl}
     improvement & at cost & cost diff & minute per euro & scenario \\
     1.500 & 1.000 & 1.000 & 1.500 & bicycle \\
     1.000 & 2.200 & 2.200 & 0.455 & public transport \\
    \end{tabular}
  \end{center}
\end{table}


The 90\% quantile Pareto Front shown in Figure \ref{fig:90_quantile_time_per_cost} with its steps shown in Table \ref{tab:differences_in_90_quantile_pareto_front} shows a similar pattern to the 75\% quantile Pareto front.
The significant difference is that the increase at \euro{2.20} for public transport scenarios is larger than the increase at 1 euro for bicycle scenarios.
More precisely, while bicycle sharing is more effective in decreasing the 15-minute city metric on average and for the 75\% most accessible regions, public transport is more effective than bicycle sharing for the 10\% least accessible areas.
We should note that even though the improvement in the public transport scenario is larger than in the other quantiles we examined, it is still less cost-effective than the improvement in the bicycle-sharing scenario.


\begin{table}
  \caption{Steps in 90\% Quantile Pareto Front}
  \label{tab:differences_in_90_quantile_pareto_front}
  \begin{center}
    \begin{tabular}{lrrrrl}
     improvement & at cost & cost diff & minute per euro & scenario \\
     3.000 & 2.200 & 2.200 & 1.364 & public transport \\
     2.000 & 1.000 & 1.000 & 2.000 & bicycle \\
     0.033 & 3.200 & 1.000 & 0.033 & public transport \\
    \end{tabular}
  \end{center}
\end{table}


\subsection{Uncertainty in Sub-scenarios}
\label{subsec:uncertainty_subscenarios}

As some of our input data is subject to uncertainties, we need to investigate the effects of this uncertainty to establish the robustness of our results.

First, we are going to look at the average standard deviation of the optimal value for the X-minute city metric in Table \ref{tab:average_standard_deviation_of_optimal_value_for_x_minute_city_metric}, which effectively shows the standard deviation of the values in Table \ref{tab:optimal_x_minute_city_metric}.
Note that we only display the average standard deviations of the bicycle, public transport, and combined scenarios, as those are the ones with uncertainty.

\begin{table}
  \caption{Average Standard Deviation of Optimal Value for X-Minute City Metric}
  \label{tab:average_standard_deviation_of_optimal_value_for_x_minute_city_metric}
  \begin{center}
    \begin{tabular}{lrrrrrrr}
     & mean & min & 25\% & 50\% & 75\% & max & CV \\
    scenario &  &  &  &  &  &  &  \\
    bicycle & 1.16 & 0.00 & 0.00 & 0.50 & 1.73 & 13.15 & 0.093403 \\
    bicycle public transport & 0.94 & 0.00 & 0.00 & 0.74 & 1.48 & 6.73 & 0.082027 \\
    public transport & 0.27 & 0.00 & 0.00 & 0.00 & 0.00 & 8.66 & 0.021151 \\
    \end{tabular}
  \end{center}
\end{table}


The mean average standard deviation for bicycle scenarios is around a minute, while it is 0.27 for the public transport scenario.
We can also see that for the bicycle-related scenarios, the uncertainty does not affect the 25\% most accessible hexagons, while for public transport, the 75\% most accessible hexagons are not affected.
In addition, outliers exist with more than 10 minutes of deviation for the pure bicycle scenario and more than 5 minutes for the public transport-related scenarios.
Relating the standard deviation to the mean, we also calculated the Coefficient of Variation (CV) in the table, which is calculated as follows:
$$ CV = \frac{\sigma}{\mu} $$
where $\mu$ is the mean and $\sigma$ is the standard deviation.
We see that it is approximately 9\% for the bicycle-related scenarios and 2\% for the public transport scenario.

To further investigate the effects of uncertainty on a more granular level, we plot the best and worst case distribution of the optimal X-minute city for each hexagon in Figure \ref{fig:best_and_worst_case_of_optimal_time_for_each_hexagon}.
These plots are essentially the upper and lower bounds of the graph, as seen in Figure \ref{fig:optimal_x_minute_city_metric}.
In addition, we've added a line at the 15-minute mark to better relate the results in the context of the 15-minute city.
The best and worst-case values are calculated using the scenario that achieves the best X-minute city metric for a given hexagon.
\begin{figure}
  \begin{center}
    \includegraphics[width=0.95\textwidth]{Figures/results/uncertainty/optimal_best_worst_case}
  \end{center}
  \caption{Best and Worst Case of Optimal Time for each Hexagon}
  \label{fig:best_and_worst_case_of_optimal_time_for_each_hexagon}
\end{figure}
First, we see that the variation for bicycles is spread out over almost all hexagons, in comparison to public transport where the variation only really begins to happen after the 15-minute mark.
For the combined scenario, we see the expected: The variances of the public transport scenario and the bicycle scenario add up.

\subsection{Impact Of Sustainable Modes on 15-Minute Metric}
\label{subsec:impact_of_sustainable_modes_on_15_minute_metric}

To analyze the impact of sustainable modes of travel on the 15-minute city metric, we first uncover the problematic areas in which the X-minute city metric is above 15 minutes for the walking mode.
We then analyze how sustainable modes of travel can help reduce the X-minute city metric in those areas below 15 minutes.

In total, we find 552 hexagons, which have a walking time of more than 15 minutes to reach all categories, which is 30.98\% of all hexagons.
\begin{table}[h]
  \centering
  \begin{tabular}{|l|l|}
    \hline
    \textbf{Category}                                          & \textbf{Data}                \\ \hline
    Only bicycle below 15 mins                                 & 72 (13.04\%)                 \\ \hline
    Only public transport below 15 mins                        & 59 (10.69\%)                 \\ \hline
    Both bicycle and public transport below 15 mins            & 41 (7.43\%)                  \\ \hline
    Combined mode below 15 mins                                & 10 (1.81\%)                  \\ \hline
    Not reachable by sustainable modes below 15 mins           & 370 (67.03\%)                \\ \hline
  \end{tabular}
  \caption{Impact of Sustainable Modes on Reducing Walking Time Above 15 Minutes}
  \label{table:hexagons_with_walking_time_above_15_minutes}
\end{table}
Table \ref{table:hexagons_with_walking_time_above_15_minutes} presents the distribution of hexagons with a walking time above 15 minutes and how sustainable modes of transport can fix those hexagons.
By fixing a hexagon, we mean that residents in the hexagon cannot reach all necessities in under 15 minutes by walking, but they can make it in under 15 minutes by some other mode of transport.
A significant portion of these areas, amounting to 67.03\%, cannot be reached within 15 minutes using sustainable modes with the current state of infrastructure. 
Conversely, the data indicates that for 13.04\% of these hexagons, only bicycles can reduce travel time to under 15 minutes, while only public transport can achieve this for 10.69\% of the hexagons. 
7.43\% of hexagons are reachable with either one of bicycles or public transport, while an additional 1.81\% of hexagons are only accessible within this time frame when combining both modes. 


Next, we visualize these problematic areas spatially.
Figure \ref{fig:problematic_hexagons} displays hexagons in green where necessities can be reached within a 15-minute walk, in yellow where they are only accessible within 15 minutes using any sustainable transport, and in red where necessities are not reachable within this 15-minute timeframe.
\begin{figure}
  \begin{center}
    \includegraphics[width=0.50\textwidth]{Figures/results/problematic_hexagons/problematic_hexagons}
  \end{center}
  \caption{Unfixable, Fixable and Unproblematic Hexagons on a Map}
  \label{fig:problematic_hexagons}
\end{figure}
We see that in the center of Cologne, almost all hexagons qualify as 15-minute city hexagons just by walking alone.
At the city's border, we see a ring of hexagons that are only valid 15-minute hexagons through additional modes of transport.
Most of the unfixable hexagons lie in the city's suburbs, often appearing in larger groups.

Next, we look at the hexagons previously colored yellow, namely those where bicycles and public transport or a combination of both can decrease the 15-minute city metric below 15 minutes.
Figure \ref{fig:fixable_hexagons} illustrates hexagons representing areas that qualify as 15-minute cities via public transport in yellow, those that qualify through bicycle sharing in orange, and areas that meet the 15-minute city criteria through either mode in green.
\begin{figure}
  \begin{center}
    \includegraphics[width=0.70\textwidth]{Figures/results/problematic_hexagons/fixable_hexagons}
  \end{center}
  \caption{Fixable Hexagons by Mode}
  \label{fig:fixable_hexagons}
\end{figure}
The data indicates a modest trend where hexagons that achieve 15-minute city criteria solely through bicycle sharing (marked in orange) tend to be nearer to the city center than those that achieve this criterion solely via public transport.

The positions of outer clusters of fixable hexagons correlate directly with the locations of bicycles and public transport stops.
Figure \ref{fig:fixable_hexagons_examples} shows four zoomed-in excerpts from Figure \ref{fig:fixable_hexagons}, where we've added the location of public transport stops and bicycles.
\begin{figure}
     \centering
     \begin{subfigure}[b]{0.45\textwidth}
         \centering
         \includegraphics[width=\textwidth]{Figures/results/problematic_hexagons/example_1.png}
     \end{subfigure}
     \hfill
     \begin{subfigure}[b]{0.45\textwidth}
         \centering
         \includegraphics[width=\textwidth]{Figures/results/problematic_hexagons/example_2.png}
     \end{subfigure}
     \hfill
     \begin{subfigure}[b]{0.45\textwidth}
         \centering
         \includegraphics[width=\textwidth]{Figures/results/problematic_hexagons/example_3.png}
     \end{subfigure}
     \hfill
     \begin{subfigure}[b]{0.45\textwidth}
         \centering
         \includegraphics[width=\textwidth]{Figures/results/problematic_hexagons/example_4.png}
     \end{subfigure}
     \caption{Examples of Fixable Hexagons}
        \label{fig:fixable_hexagons_examples}
\end{figure}
Public Transport stops are visualized as yellow circles, while bicycles are visualized as orange circles.
We notice that hexagons fixed by bicycle sharing are always near bicycles.
In the same way, hexagons fixed better by public transport are always close to public transport stops. 
However, being close to bike stations seems to have a more significant effect than being near public transport stops.

Figure \ref{fig:only_unfixable_hexagons} shows all hexagons that are not 15-minute hexagons by any sustainable mode of transport.
Figure \ref{fig:unfixable_with_bicycles} and \ref{fig:unfixable_with_stops} show the same map but with additional bicycle locations and public transport stop locations, respectively.
\begin{figure}
     \centering
     \begin{subfigure}[b]{0.30\textwidth}
         \centering
         \includegraphics[width=\textwidth]{Figures/results/problematic_hexagons/unfixable.png}
         \caption{Only Unfixable Hexagons}
         \label{fig:only_unfixable_hexagons}
     \end{subfigure}
     \hfill
     \begin{subfigure}[b]{0.30\textwidth}
         \centering
         \includegraphics[width=\textwidth]{Figures/results/problematic_hexagons/unfixable_with_bicycles.png}
         \caption{With All Bicycle Locations}
         \label{fig:unfixable_with_bicycles}
     \end{subfigure}
     \hfill
     \begin{subfigure}[b]{0.30\textwidth}
         \centering
         \includegraphics[width=\textwidth]{Figures/results/problematic_hexagons/unfixable_with_stops.png}
         \caption{With Public Transport Stops}
         \label{fig:unfixable_with_stops}
     \end{subfigure}
     \hfill
     \caption{Unfixable Hexagons}
     \label{fig:unfixable_hexagons}
\end{figure}
We can observe that the unfixable hexagons mostly don't contain any bicycles and have a larger distance to the nearest bicycle.
The same cannot be said for public transport stops, as public transport stops are often directly inside the unfixable hexagons.

Figure \ref{fig:combined_hexagons} shows all hexagons that only become 15-minute city hexagons in the combined scenario when public transport and bicycle sharing are used simultaneously.
As already seen in Table \ref{table:hexagons_with_walking_time_above_15_minutes}, this only concerns less than 2\% of all hexagons that are not already 15-minute city hexagons through walking alone.
More than half of those (7 out of 10) are located in the southern district of Weiß.
\begin{figure}
  \begin{center}
    \includegraphics[width=0.50\textwidth]{Figures/results/problematic_hexagons/combined_hexagons}
  \end{center}
  \caption{Hexagons Fixable by Combined Mode}
  \label{fig:combined_hexagons}
\end{figure}


\subsection{Monthly Costs Per Scenario And Hexagon}
This section is not ready.

A prevalent measure to incentivize sustainable modes of transport are monthly tickets or subscriptions.
To measure whether the costs of these subscriptions are worth it, we will calculate the monthly cost incurred by the trips to all necessities.
To do so, we first collected how often people visited each category we defined earlier.
Table \ref{tab:monthly_visits} shows the monthly number of visits per category.

\begin{table}
  \caption{Number of Monthly Visits per Category}
  \label{tab:monthly_visits}
  \begin{center}
    \begin{tabular}[c]{l|l}
      category & monthly visits \\
      \hline
      groceries & 12 \\
      education & 20 \\
      health & 0.42 \\
      banks & 9 \\
      parks & 2.4 \\
      sustenance & 6.12 \\
      shops & 4 \\
      \hline
    \end{tabular}
  \end{center}
\end{table}

The derivation of these numbers can be found in Appendix \ref{app:monthly_visits_per_category}

To understand and compare the usual monthly costs caused by traveling with different modes of transport, we first establish two time-based benchmarks to compare the cost incurred.
The first benchmark focuses on the costs incurred when reaching the nearest POI of a given category within a 15-minute timeframe. 
Conversely, the second benchmark assesses the costs for a similar journey but within a more constrained 10-minute limit.
However, to compare these costs, a journey in that time has to be possible, which is not always the case.


\begin{figure}
  \centering
  \begin{subfigure}[b]{0.45\textwidth}
    \centering
    \includegraphics[width=\textwidth]{Figures/results/monthly_costs/percentage_inf_10.png}
    \caption{10 Minutes}
    \label{fig:percentage_inf_10}
  \end{subfigure}
  \hfill
  \begin{subfigure}[b]{0.45\textwidth}
    \centering
    \includegraphics[width=\textwidth]{Figures/results/monthly_costs/percentage_inf_15.png}
    \caption{15 Minutes}
    \label{fig:percentage_inf_15}
  \end{subfigure}
  \caption{Impossible Sub-X-Minute Journeys for each Hexagon and Scenario}
  \label{fig:percentage_inf_x}
\end{figure}

Therefore, we show how often a journey is possible within the given timeframe in Figure \ref{fig:percentage_inf_x}.
The combined scenario of bicycle sharing and public transport fails the least, followed by public transport and bicycle sharing.
In the 15-minute benchmark, bicycle sharing and public transport perform almost equally well, while in the 10-minute benchmark, bicycle sharing performs better than public transport.

Category-wise, banks seem to be the least accessible category, followed by parks and health.
Sustenance, education, grocery, and shops all seem similarly accessible.

As seen in the previous Figure, the number of impossible journeys differs across scenarios.
This, again, makes it hard to compare the costs across scenarios.
Therefore, we will only look at hexagons and category combinations where a journey is possible within the given timeframe for all scenarios.
In addition, we also filter out hexagon category combinations where the cost is zero, i.e. where walking suffices, as those are not interesting for our cost analysis.

\begin{table}
  \caption{Monthly costs per scenario (<15 minutes)}
  \label{tab:monthly_costs_per_scenario_15}
  \begin{center}
    \begin{tabular}{lrrrrrrr}
     & mean & std & min & 25\% & 50\% & 75\% & max \\
    Scenario &  &  &  &  &  &  &  \\
    Bicycle & 5.79 & 5.95 & 0.42 & 0.42 & 4.00 & 9.00 & 29.00 \\
    Combined & 5.79 & 5.95 & 0.42 & 0.42 & 4.00 & 9.00 & 29.00 \\
    Public Transport & 12.93 & 13.26 & 0.93 & 0.93 & 8.80 & 19.80 & 63.80 \\
    \end{tabular}
  \end{center}
\end{table}


\begin{table}
  \caption{Monthly costs per scenario (<10 minutes)}
  \label{tab:monthly_costs_per_scenario_10}
  \begin{center}
    \begin{tabular}{lrrrrrrr}
     & mean & std & min & 25\% & 50\% & 75\% & max \\
    Scenario &  &  &  &  &  &  &  \\
    Bicycle & 10.49 & 7.72 & 0.42 & 2.42 & 9.00 & 20.00 & 24.54 \\
    Combined & 10.49 & 7.72 & 0.42 & 2.42 & 9.00 & 20.00 & 24.54 \\
    Public Transport & 23.08 & 16.98 & 0.93 & 5.32 & 19.80 & 44.00 & 53.98 \\
    \end{tabular}
  \end{center}
\end{table}

Table \ref{tab:monthly_costs_per_scenario_15} shows the average monthly costs to reach all categories within 15 minutes for each scenario, and Table \ref{tab:monthly_costs_per_scenario_10} shows the same for 10 minutes.
As we can see, the total cost averages at \euro{5.80} for the bicycle and combined scenario and \euro{12.90} for the public transport scenario in the 15-minute benchmark.
The cost reaches a maximum of \euro{29} for the bicycle and combined scenario and \euro{63.8} for the public transport scenario, while the 25\% most expensive hexagons require residents to pay 9\euro{} for the bicycle and combined scenario and 19.80 for the public transport scenario.
Note that the bicycle and combined scenario always have the same cost.
In the 10-minute benchmark, the average cost is \euro{10.5} for the bicycle and combined scenario and \euro{23.1} for the public transport scenario.
Here, the cost reaches a maximum of \euro{24.54} for the bicycle and combined scenario and \euro{53.98} for the public transport scenario, while the 25\% most expensive hexagons require residents to pay \euro{20} for the bicycle and combined scenario and \euro{44} for the public transport scenario.

\begin{table}
  \caption{Monthly costs for cars}
  \label{tab:monthly_costs_for_cars}
  \begin{center}
    \begin{tabular}{lrrrrrrr}
     & mean & std & min & 25\% & 50\% & 75\% & max \\
    Car (15 Minutes) & 1.79 & 1.69 & 0.00 & 0.46 & 1.71 & 1.79 & 7.00 \\
    Car (10 Minutes) & 2.87 & 2.89 & 0.00 & 0.46 & 1.79 & 3.91 & 10.88 \\
    \end{tabular}
  \end{center}
\end{table}

Next, we will also look at the monthly costs for the car scenario in the identical hexagons we looked at before in Figure \ref{tab:monthly_costs_for_cars}.
This allows us to compare the cost of cars with other modes of transport.
Remember that the cost of 19ct/min tries to capture fuel costs, repair costs, insurance, and tax, but not acquisition costs.

\clearpage
\section{Discussion}
\label{sec:discussion}

\subsection{Interpretation of Results}
\subsubsection{Substitutional Effects and Expanding of Bicycle Fleet to Less Accessible Areas}

% both bicycles and pt improve
Analyzing the mean and quantiles of the optimal X-minute city in Table \ref{tab:optimal_x_minute_city_metric} for each hexagon and scenario, revealed that both bicycle sharing and public transport, when added to another scenario, can decrease the time it takes to reach all necessities.
The average improvement that comes from adding bicycle to the mix varies between 1:16 minute and 1:36 and the average improvement that comes from adding public transport to the mix varies between 0:56 and 1:18.
Therefore, we can say that both methods are effective in improving accessibility over walking.


\paragraph{Public Transport Effectiveness}
% pt for remote areas of low accessiblity 
Various findings lead us to the conclusion that while public transport effective in improving accessibility, the most benefit is generated in remote suburban areas, that have a low density of POIs and that are located near to stops connected to high frequency public transport lines.
In addition, we find that public transport, compared to bicycle sharing is more costly.

In Figure \ref{fig:mean_time_per_cost} we can clearly see that when users are able to afford a short public transport trip (cost of 2.20), their accessibility to POIs increases.

% pt for areas of low accessiblity 
Firstly, the improvement seen by adding public transport mainly comes from areas with low accessibility as seen in Table \ref{tab:optimal_x_minute_city_metric} and Figure \ref{fig:optimal_x_minute_city_metric}.
These areas, or hexagons, are likely far from POIs, and we suspect that  public transport helps by covering these longer distances. 
However, in areas where it's already easy to get to these places, adding public transport might not make a big difference. 
This is because using public transport often involves extra steps: walking to the bus or train stop and waiting for it to arrive. 
How much this extra time matters depends on how long the whole trip is.
For short trips, the time spent getting to and waiting for public transport could be a large part of the travel time, making it less useful. 
But for longer trips, especially in the areas that were hard to reach before, this extra time is a smaller part of the journey. 
This makes public transport more beneficial for these longer trips. 

% pt actually better than bicycle for 20% worst
Secondly, in Figure \ref{fig:optimal_x_minute_city_metric} we see the full distribution of the optimal X-minute city metric for each scenario.
Notably, while public transport is worse than bicycle for the 80\% most accessible hexagons it shows a distinct advantage over bicycle sharing beyond the 80\% quantile, facilitating faster access in less accessible areas. 
The reason for this change in effectiveness between public transport and bicycle sharing is linked to how bicycle sharing is set up in Cologne. 
With the majority of bicycles available in the city center's "Flex-Zone", suburban areas have fewer bicycles as they can only be found at stations.
Consequently, the least accessible hexagons, typically situated in suburban regions, experience low to no availability of bicycles, which explains the superiority of public transport in these areas. 

% rath/neumar is reachable better by pt
This hypothesis is further supported when comparing the optimal X-minute city metric spatially between public transport and bicycle sharing, as seen in Figure \ref{fig:bicycle_optimal_map} and \ref{fig:public_transport_optimal_map}.
We see the district of Rath/Neumar in the east, which shows low accessibility in general.
However, we can clearly see that public transport has an advantage over bicycle sharing in this remote area.

Similarly, when looking at Table \ref{table:hexagons_with_walking_time_above_15_minutes} we see that public transport is able to make 10\% of hexagons that are not valid in terms of the 15-minute city by walking alone, valid.
The magnitude is  comparable to that of bicycle sharing, however, when looking at the spatial distribution of those hexagons in Figure \ref{fig:fixable_hexagons}, we see that the yellow hexagons, which are those that, become valid in terms of the 15-minute city only through public transport, are mostly located outside the city in remote areas.

% pt only better at 90% pareto front
In addition, when investigating the cost and X-minute city metric Pareto fronts in Figure \ref{fig:quantile_time_per_cost}, we see that public transport is only able to yield larger improvements than bicycle sharing when considering only the 10\% worst accessible hexagons.

% cost stuff
Table \ref{tab:required_cost} and Figure \ref{fig:maximum_required_cost_for_x_minute_city} show that public transport's advantage at the least accessible hexagons comes at a cost.
As soon, as the benefits of public transport manifest, the cost also increases to 2.20€, which is obvious as public transport rides are always charged.
We can see, however, that is really rarely necessary to travel more than four stops, as the cost for that (3.20€) is only reached at the very end. 
The three- or four-step increases to the cost of 2.20€ and 3.20€ can be explained by the different sub-scenarios.
In some scenarios it might be only beneficial to use public transport later.

Looking at Figure \ref{fig:maximum_required_cost_for_x_minute_city} also clearly reveals the spatial usage pattern of public transport.
As already conjectured previously, we see that mostly public transport is used in remote locations outside the city, that we know have lower accessibility in general.

The single hexagons inside the cities seen in Figure \ref{fig:maximum_required_cost_for_x_minute_city} are most likely located very close to a public transport stop, enabling to use the public transport system without any loss of time.
Inside the city it seems to only be beneficial to use the public transport system to reach necessities when living near a stop.
However, outside the city the larger groups of hexagons indicate that using the public transport system is often faster than walking to the necessities, eventhough walking to the next stop requires some time.
This may be, because the density of the POIs is lower outside the city.

% comparison of usefulness of short and long trips
With the help of the cost and X-minute city metric Pareto fronts, we are able to evaluate the usefulness and cost-efficiency of short trips (those that travel no more than 4 stops) and long trips (those that travel more than 4 stops).
Figure \ref{fig:mean_time_per_cost} and Table \ref{tab:differences_in_mean_pareto_front} show that the improvement caused by the short trip tickets is on average 1.28 minutes, compared to the 0.074 minutes of long trip tickets, and therefore almost 20 times larger.
Different from what we previously assumed, long distance trips actually don't bring a lot of benefit.
However, we should note that just because it is not beneficial to travel more than four stops, this does not mean that trips associated with four stops or fewer don't travel long distances.
Especially in suburban areas, where public transport stops are more sparse than in the city center, it might very well be that trips with four stops or fewer still travel multiple kilometers.
When investigating the 25\% quantile and 75\% quantile Pareto fronts in Figure \ref{fig:quantile_time_per_cost}, there is no benefit of long distance tickets displayed.
Only when investigating the 90\% quantile Pareto front, the benefit is visible.
This indicates that long distance trips are only used for the absolute worst accessible hexagons.
% END 

\paragraph{Bicycle Sharing Effectiveness}
% intro
The implementation of bicycle sharing has shown to enhance accessibility in a broad spectrum of urban areas. 
This improvement is not limited to less connected zones but is also evident in areas already well-served by existing transport networks.
Additionally, bicycle sharing has shown to be more cost-efficient than public transport.
Bicycle sharing is mostly only effective in areas, where "Flex-Zones" are and it is highly dependent on the allocation of bicycles.

% bicycles improve in general
In Figure \ref{fig:mean_time_per_cost} we can clearly see that when users are able to afford a bicycle for 15 minutes (cost of 1€), their accessibility to POIs increases drastically.


As detailed in Table \ref{tab:optimal_x_minute_city_metric} and Figure \ref{fig:optimal_x_minute_city_metric}, the introduction of bicycle sharing yields benefits across almost all hexagons, offering a more uniform impact compared to public transport. 
It is particularly notable that bicycle sharing also yields improvements in already well-accessible areas. 
We think that this is due to the fact that bicycles have a lower overhead which in turn contributes significantly to their practicality in urban settings.
Firstly, the higher density of bicycle access points compared to public transport stops inherently reduces the initial distance required to access a mode of transport, which  facilitates quicker access to the transport system. 
Secondly, bicycles eliminate the waiting period often associated with public transport schedules, which means that once a user reaches a bicycle, they can immediately start their journey. 
This immediacy and ease of access render bicycles an effective solution for a more general scope than public transport.

Further, the data in Figure \ref{fig:optimal_x_minute_city_metric} reveals that in more accessible hexagons, combining bicycles with public transport offers greater advantages over using public transport alone. 
Interestingly, looking at least accessible hexagons, the disparity between these two scenarios narrows. 
This observation implies that in areas with very low accessibility, which are most likely remote areas, the addition of bicycles does not significantly enhance accessibility.
This trend is likely due to the limited availability of bicycles in these less accessible areas, underscoring the importance of equitable distribution in bicycle sharing systems.

We already hypothesized that the low utility for bicycle sharing in areas with low accessibility is linked to the low availability of bicycles in suburban areas, as the majority of bicycles is located in the "Flex-Zone".
This hypothesis gains further support when contrasting Figures \ref{fig:bicycle_optimal_map} and \ref{fig:walking_optimal_map}. 
Here, an improvement is observed in the bicycle scenario compared to walking, particularly within the "Flex-Zone" as shown in Figure \ref{fig:flex_zones}, which underscores the impact of the zone on accessibility patterns.

\begin{figure}
  \begin{center}
    \includegraphics[width=0.50\textwidth]{Figures/discussion/flex_zones.png}
  \end{center}
  \caption{Next Bike's Flex Zones}
  \label{fig:flex_zones}
\end{figure}

% pt only better at 90% pareto front
Investigating the cost and X-minute city metric Pareto fronts in Figures \ref{fig:mean_time_per_cost} and \ref{fig:quantile_time_per_cost}, as well as, the corresponding values of improvement in Tables \ref{tab:differences_in_mean_pareto_front}, \ref{tab:differences_in_75_quantile_pareto_front} and \ref{tab:differences_in_90_quantile_pareto_front} shows that using a bicycle for 15 minutes always yields larger time gains than public transport except for the 10\% worst accessible hexagons.
This is a clear indicator for the superiority of bicycle sharing over public transport in terms of accessibility improvements, especially, when considering that the bicycle sharing infrastructure does not cover all of the considered area.


% cost stuff
Table \ref{tab:required_cost} and Figure \ref{fig:maximum_required_cost_for_x_minute_city} clearly indicate that bicycle sharing never costs more than public transport, if both are used.
We can also deduct this from the price of both and the fact that bicycles are never used more than 15 minutes.
We now that bicycles are never used for more than 15 minutes as the maximum cost, as seen in Table \ref{tab:required_cost} is 1.00€, which is the price for a 15-minute ride.
Meanwhile, the shortest possible public transport trip costs 2.20€, therefore as soon as public transport is used it always costs more than bicycles.
Nevertheless, this already shows that bicycle sharing is cheaper than public transport. 

In addition, knowing that bicycles never are used for more than 15-minute tells us that at any location where bicycles are available, Cologne is a 15-minute city.
This is a strong indicator that bicycle sharing is able to make regions 15-minute city regions.

Table \ref{tab:differences_in_mean_pareto_front} shows the average improvements of the X-minute city metric, at the cost points where public transport and bicycle sharing are used.
We see that bicycle sharing is on average more than 3 times more cost-efficient than public transport.
Further, investigating the differences at the 75\% and 90\% quantile Pareto fronts in Table \ref{tab:differences_in_75_quantile_pareto_front} and \ref{tab:differences_in_90_quantile_pareto_front}, shows that bicycle sharing always is more cost-efficient than public transport.
This again proves that expanding bicycle sharing zones to suburban areas might be more beneficial to residents, than improving public transport.

Similarly, when looking at Table \ref{table:hexagons_with_walking_time_above_15_minutes} we see that bicycle sharing is able to make 13\% of hexagons that are not valid in terms of the 15-minute city by walking alone, valid, which is more than public transport.
This shows that bicycle sharing is more effective than public transport.

Looking at the spatial distribution of those hexagons in Figure \ref{fig:fixable_hexagons}, we see that the orange hexagons, which are those that, become valid in terms of the 15-minute city only through bicycle sharing, are mostly located at the city's border, where the "Flex-Zone" is still active.
This again shows that the effectiveness of  bicycle sharing highly depends on the flex zone.


%END

% not sure where to put this
Another difference between bicycles and public transport is that they differ notably in their speed and route flexibility. 
Public transport typically travels faster than bicycles, which offers a clear advantage for covering long distances. 
This speed is particularly beneficial when some POIs are far away, as public transport can effectively bridge these larger gaps. 
However, the fixed routes of public transport mean that users may not always disembark close to their desired POI. 
In contrast, bicycles offer the flexibility of route choice, allowing users to tailor their journey to arrive nearer to the POI. 
This also aligns with the finding that public transport tends to be most effective for the least accessible hexagons, where distance to POIs is a significant factor.

% not sure where to put this
In general, bicycle sharing brings larger improvements than public transport, while remaining more cost-effective.
The improvements of bicycle sharing are effective for the most accessible regions, while the improvements of public transport are not.
This shows that extending the bicycle sharing system might be a good way to improve the accessibility in Cologne.

However, for the least accessible regions, public transport is more effective than bicycle sharing.
This most likely means that there are regions in the region of Cologne, where some categories of POIs are not present.
We suspect that those regions are characterized by low accessibility and low or no availability of bicycles.
In those regions public transport becomes more effective than bicycles, as it essentially allows to relocate to better areas.
In the spirit of the 15-minute city, it might be more desirable to fix the sparsity of POIs than to substitute public transport.
% end


\paragraph{Bicycle Sharing and Public Transport - Substitutes or Complements?}
% introduction of substitutional effects and complement I & II
As already mentioned we found clear evidence that bicycle sharing and public transport have a positive effect on accessibility according to the X-minute city metric.
It is, however, not clear whether these modes are substitutional to each other, meaning that one mode is able to compensate the other, or whether they are complements.
We potentially could observe two sorts of complementary effects.
The first (complement I) would be that bicycle sharing and public transport each are effective under different circumstances and in different areas.
If that's the case, one cannot replace the other, and each is necessary under specific circumstances.
The second effect (complement II) would be that sometimes bicycles sharing and public transport used together is necessary to improve vehicle sharing.
If that's the case, there both would be necessary to be employed together in order to achieve the highest accessibility.
Both effects can be described as complementary effects, and they are not mutually exclusive, which means that we can potentially observe both.


Table \ref{tab:optimal_x_minute_city_metric} shows that the combined scenario on average yields better accessibility than bicycle sharing and public transport alone.
This suggests that either complement I or complement II must be present.

% TODO: complement I
We see evidence for complement I in the observation that bicycle sharing is most effective for the 80\% more accessible hexagons, while public transport is mostly only effective for the 20\% least accessible hexagons.

Table \ref{table:hexagons_with_walking_time_above_15_minutes} shows how many hexagons in which people are not able to reach all necessities in 15 minutes by walking can reach all necessities in under 15 minute by public transport or bicycle sharing.
We see that bicycles sharing, and public transport alone are roughly of equal importance, as they fix 13\% and 11\% of hexagons, respectively.
In addition, a lot more hexagons are fixed by only one of them (24\%) than by either of them (7\%).
This again suggests that complement I is present.


% TODO: complement II
Table \ref{table:hexagons_with_walking_time_above_15_minutes} also shows that the combined scenario only fixes around 2\% and is therefore not very impactful.
Nevertheless, we prove that the effect of complement II is present.

% TODO: implications of complement I & II
% implications
The presence of complement I should make us aware that the effectiveness of either mode is highly dependent on the spatial circumstances of the region.
Some regions might benefit more from bicycle sharing, while others benefit more from public transport.
It is therefore, necessary to analyze each region separately, to maximize the positive effect of sustainable modes of travel.
% implications END
%END

% not sure where to put this "conclusion"
Our findings lead to the conclusion that enhancing bicycle availability in areas with low accessibility could yield significant benefits.
Additionally, the observed dynamics suggest that bicycles are more advantageous in areas with low to medium accessibility, while public transport predominantly benefits the least accessible areas.
This again indicates that bicycles and public transport serve complementary, rather than substitutable, roles in urban mobility networks.
Therefore, removing one mode of transportation cannot be effectively compensated for by simply increasing the other, as each serves distinct and crucial functions in addressing different aspects of urban accessibility.


\subsubsection{Specific Recommendations for Cologne}

% Merkenich & rhine bridge
In Figure \ref{fig:optimal_map} we observed that Merkenich, which is located at the other side of the Rhine next to Leverkusen, is one of the least accessible regions in Cologne.
From Figure ???, we now that there are plenty of POIs for all categories in Leverkusen, which are very close to the problematic region.
This indicates a lack of mobility crossing the Rhine river.
As a bridge is already existing, we suggest providing a frequent public transport line to improve the accessibility in Merkenich.

% Rath/Neumar
In Figure \ref{fig:public_transport_optimal_map} we see that while the district of Rath/Neumar shows bad accessibility for all three scenarios, the accessibility in the public transport is better than that of walking and bicycle sharing.
This is not surprising, as this region is quite far away from POIs, which makes walking very slow.
Also, the region is not in NextBike's flex zone, which leads to a low availability of bicycles, explaining why the bicycle sharing scenario is as bad as walking here.
However, the city train line 9 runs through this region, which leads to a higher accessibility by public transport.

This fact shows us that, a high frequency public transport line from the city center to less accessible area can significantly improve accessibility.
It might be beneficial to create such lines to other less accessible regions in the north and south of Cologne's suburban areas.


\subsubsection{Time Gains per Euro}

Table \ref{tab:differences_in_mean_pareto_front} shows how a minute of saved time costs for cycling and public transport, respectively.
The most cost-efficient variant, bicycle sharing, enables users to save 1.68 minutes per euro, which might not be worth it for most people.
The 1.68 minutes per euro spent, in other words means that to save a minute of time people have to spend around 62 cents.
Comparing this to the minimum wage in Germany of 12 euro per hour in 2022 \cite{federalstatisticalofficegermanyMinimumWages}, which is 20 cents per minute, shows that it takes around three minutes to make enough money to save one minute, which obviously is not worth it.
This means that currently on average the sustainable modes of travel are too expensive.

Even when looking at the least accessible regions (90\% quantile) separately in Figure \ref{fig:90_quantile_time_per_cost}, where we have the largest improvement possible, the most cost-efficient mode of transport, which is bicycle, only yields an improvement of two minutes per euro, which again does not result in a net positive for people who earn the minimum wage.

% TODO: more interpretation of Differences in  Pareto Front tables


\subsubsection{Uncertainty}

The standard deviation of the average time it takes to reach all necessities for bicycles is 1 minute and 10 second. 
Relating this to the improvement compared to walking of 1 minute and 38 seconds, we see that the uncertainty is quite large.
This shows us that the placement of bicycles significantly impacts the effectiveness of the bicycle sharing scenario.

It is interesting to note that bicycles suffer more from uncertainty than public transport, as public transport only has a standard deviation of 16 seconds.
Relating the standard deviation to the improvement compared to walking of 1 minute and 28 seconds, it it by far not as impactful as with bicycles.
However, the standard deviation strongly depends on the choices of our sub-scenarios. 
For public transport we tried 08:00, 12:00 and 18:00.
We suspect that the variance for public transport would increase more drastically and eventually surpass the one of bicycle sharing if we choose more unusual times like midnight.
In addition, the schedule of public transport is often the same for full hours.
We could get a more realistic picture, if we would add more times with non-zero minutes to our sub scenarios.

We should also note that the comparison of variances should be taken with caution as the method for selecting the sub-scenarios differ per scenario.
For bicycle sharing we employed a clustering method, while for public transport we made a qualitative choice. 

\subsubsection{Potential of Sustainable Modes To Transform Cities To 15-Minute Cities}

We know that approximately 69\% of all considered hexagons and with that 69\% of all residential in the administrative district Cologne ("Stadtkreis Köln") are 15-minute city by walking, which shows us that Cologne largely already has an excellent accessibility.

Table \ref{table:hexagons_with_walking_time_above_15_minutes} shows that a majority (67\%) of the hexagons that are currently not 15-minute city regions are not fixed by any of the sustainable modes of travel, which shows that the impact of public transport and bicycle is limited.
Nevertheless, the impact of bicycle and public transport on making regions valid in terms of the 15-minute city is not negligible, as those modes of transport still achieve to fix 33\% of hexagons that are not yet valid in terms of the 15-minute city by walking alone.

Figure \ref{fig:problematic_hexagons} shows hexagons that are valid in terms of the 15-minute city by walking in green, those that are valid only by the addition of any sustainable mode of travel in yellow, and those that are not valid through any sustainable mode of travel in red.
We suspect that the very noticeable yellow ring around the city center is in an area where there are fewer POIs, but still a high availability of bicycles, that can compensate this sparsity.
This spatially shows where sustainable modes of travel are important to compensate for the sparsity of POIs.
These areas don't provide a close enough proximity to be considered valid in terms of the 15-minute city by walking alone.
However, with bicycle sharing and public transport they are.
In Figure \ref{fig:fixable_hexagons} we see that most of the hexagon in the ring are orange or green, which means that they are either fixable by bicycle sharing or by either bicycle sharing or public transport.
This again underlines that bicycle sharing is more effective, if it is present, than public transport.
In the same figure, we can see that yellow hexagons, those that are only fixable through public transport, tend to exist in the regions that are more far away.

The unfixable regions mostly don't have bicycles near them, but they often have public transport nearby.
This suggests that bicycles are more effective than public transport to help a region become sub 15-minute.



%%%%%%%%%%%%%%%%%%%%%%%%%%%%%%%%%%%%%%%%%%%%%%%%%%%%%%%%%%%%%
%APPENDICES
%%%%%%%%%%%%%%%%%%%%%%%%%%%%%%%%%%%%%%%%%%%%%%%%%%%%%%%%%%%%%


\appendix
\renewcommand*{\thesection}{\Alph{section}}\textbf{}

% APPENDIX A
\clearpage
\section{Appendix - Overpass Query for Boundary of Cologne}
\label{app:overpass_query}
\begin{verbatim}
[out:json][timeout:50];
area["name"="Köln"]->.searchArea;
relation["boundary"="administrative"]["admin_level"="6"](area.searchArea);
out body;
>;
out skel qt;
\end{verbatim}

\section{Appendix - Pyrosm Network Filter}
\label{app:pyrosm_network_filter}

Pyrosm filters out all ways that have the following tags:

\begin{table}[h]
\centering
\caption{Driving Filter}
\begin{tabular}{|c|p{10cm}|}
\hline
\textbf{Key}         & \textbf{Values}                                                                                                             \\ \hline
area                 & yes                                                                                                                         \\ \hline
highway              & cycleway, footway, path, pedestrian, steps, track, corridor, elevator, escalator, proposed, construction, bridleway, abandoned, platform, raceway \\ \hline
motor\_vehicle       & no                                                                                                                          \\ \hline
motorcar             & no                                                                                                                          \\ \hline
service              & parking, parking\_aisle, private, emergency\_access                                                                         \\ \hline
\end{tabular}
\end{table}


\begin{table}[h]
\centering
\caption{Walking Filter}
\begin{tabular}{|c|p{10cm}|}
\hline
\textbf{Key}         & \textbf{Values}                                                                                                             \\ \hline
area                 & yes                                                                                                                         \\ \hline
highway              & cycleway, motor, proposed, construction, abandoned, platform, raceway, motorway, motorway\_link                             \\ \hline
foot                 & no                                                                                                                          \\ \hline
service              & private                                                                                                                     \\ \hline
\end{tabular}
\end{table}


\begin{table}[h]
\centering
\caption{Cycling Filter}
\begin{tabular}{|c|p{10cm}|}
\hline
\textbf{Key}         & \textbf{Values}                                                                                                             \\ \hline
area                 & yes                                                                                                                         \\ \hline
highway              & footway, steps, corridor, elevator, escalator, motor, proposed, construction, abandoned, platform, raceway, motorway, motorway\_link \\ \hline
bicycle              & no                                                                                                                          \\ \hline
service              & private                                                                                                                     \\ \hline
\end{tabular}
\end{table}


\section{Appendix - Categories and Their Corresponding OSM Tags}
\label{app:categories_and_osm_tags}

\begin{table}[ht]
\centering
\caption{Categories and Their Corresponding OSM Tags}
\label{tab:categories}
\footnotesize
\begin{tabular}{|l|l|p{10cm}|}
\hline
\textbf{Category} & \textbf{OSM Key} & \textbf{OSM Value} \\ \hline
Grocery           & shop             & alcohol, bakery, beverages, brewing supplies, butcher, cheese, chocolate, coffee, confectionery, convenience, deli, dairy, farm, frozen food, greengrocer, health food, ice-cream, pasta, pastry, seafood, spices, tea, water, supermarket, department store, general, kiosk, mall \\ \hline
Education         & amenity          & college, driving school, kindergarten, language school, music school, school, university \\ \hline
Health            & amenity          & clinic, dentist, doctors, hospital, nursing home, pharmacy, social facility \\ \hline
Banks             & amenity          & atm, bank, bureau de change, post office \\ \hline
Parks             & leisure          & park, dog park \\ \hline
Sustenance        & amenity          & restaurant, pub, bar, cafe, fast-food, food court, ice-cream, biergarten \\ \hline
Shops             & shop             & department store, general, kiosk, mall, wholesale, baby goods, bag, boutique, clothes, fabric, fashion accessories, jewelry, leather, watches, wool, charity, secondhand, variety store, beauty, chemist, cosmetics, erotic, hairdresser, hairdresser supply, hearing aids, herbalist, massage, medical supply, nutrition supplements, optician, perfumery, tattoo, agrarian, appliance, bathroom furnishing, do-it-yourself, electrical, energy, fireplace, florist, garden centre, garden furniture, fuel, glaziery, groundskeeping, hardware, houseware, locksmith, paint, security, trade, antiques, bed, candles, carpet, curtain, flooring, furniture, household linen, interior decoration, kitchen, lighting, tiles, window blind, computer, electronics, hifi, mobile phone, radio-technics, vacuum cleaner, bicycle, boat, car, car repair, car parts, caravan, fishing, golf, hunting, jet ski, military surplus, motorcycle, outdoor, scuba diving, ski, snowmobile, swimming pool, trailer, tyres, art, collector, craft, frame, games, model, music, musical instrument, photo, camera, trophy, video, videogames, anime, books, gift, lottery, newsagent, stationery, ticket, bookmaker, cannabis, copy shop, dry cleaning, e-cigarette, funeral directors, laundry, moneylender, party, pawnbroker, pet, pet grooming, pest control, pyrotechnics, religion, storage rental, tobacco, toys, travel agency, vacant, weapons, outpost \\ \hline
\end{tabular}
\normalsize
\end{table}

\section{Appendix - Experiment Module Matrix Configuration}
\label{app:experiment_module_matrix_configuration}

TODO: add module matrix configuration for each scenario here

\section{Appendix - Monthly Visits per Category}
\label{app:monthly_visits_per_category}

TODO






%%%%%%%%%%%%%%%%%%%%%%%%%%%%%%%%%%%%%%%%%%%%%%%%%%%%%%%%%%%%%
%BIBLIOGRAPHY
%%%%%%%%%%%%%%%%%%%%%%%%%%%%%%%%%%%%%%%%%%%%%%%%%%%%%%%%%%%%%

\clearpage
\renewcommand*{\thesection}{}\textbf{}

\bibliographystyle{apacite}
\bibliography{Bibliography.bib}


\end{document}
